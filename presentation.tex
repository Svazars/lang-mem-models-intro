\documentclass{beamer}

\documentclass[aspectratio=169]{beamer}

% activate me to make slides with no animation
%\documentclass[handout]{beamer}

\usepackage[warn]{mathtext}
\usepackage[T2A]{fontenc}
\usepackage[utf8]{inputenc}
\usepackage[english,russian]{babel}

\usepackage{amssymb}
\usepackage{amsmath}
\usepackage{multirow}
\usepackage{graphicx}
\usepackage{verbatim}
\usepackage{comment} 

\usepackage[cache=false]{minted}
\usepackage{listings}

\lstset{language=Java,
                basicstyle=\footnotesize\ttfamily,
                keywordstyle=\footnotesize\color{blue}\ttfamily,
}


\usepackage{adjustbox}


%%%%%%%%%%%%%%%%%%%%%%%%%%%%%%%%%  fix-lstlinebgrd.tex 
\makeatletter
\let\old@lstKV@SwitchCases\lstKV@SwitchCases
\def\lstKV@SwitchCases#1#2#3{}
\makeatother
\usepackage{lstlinebgrd}
\makeatletter
\let\lstKV@SwitchCases\old@lstKV@SwitchCases
        
\lst@Key{numbers}{none}{%
    \def\lst@PlaceNumber{\lst@linebgrd}%
    \lstKV@SwitchCases{#1}%
    {none:\\%
     left:\def\lst@PlaceNumber{\llap{\normalfont
                \lst@numberstyle{\thelstnumber}\kern\lst@numbersep}\lst@linebgrd}\\%
     right:\def\lst@PlaceNumber{\rlap{\normalfont
                \kern\linewidth \kern\lst@numbersep
                \lst@numberstyle{\thelstnumber}}\lst@linebgrd}%
    }{\PackageError{Listings}{Numbers #1 unknown}\@ehc}}
\makeatother
%%%%%%%%%%%%%%%%%%%%%%%%%%%%%%%%%


%%%%%%%%%%%%%%%%%%%%%%%%%%%%%%%%%  bListHL
\makeatletter
%%%%%%%%%%%%%%%%%%%%%%%%%%%%%%%%%%%%%%%%%%%%%%%%%%%%%%%%%%%%%%%%%%%%%%%%%%%%%%
%
% \btIfInRange{number}{range list}{TRUE}{FALSE}
%
% Test in int number <number> is element of a (comma separated) list of ranges
% (such as: {1,3-5,7,10-12,14}) and processes <TRUE> or <FALSE> respectively

\newcount\bt@rangea
\newcount\bt@rangeb

\newcommand\btIfInRange[2]{%
    \global\let\bt@inrange\@secondoftwo%
    \edef\bt@rangelist{#2}%
    \foreach \range in \bt@rangelist {%
        \afterassignment\bt@getrangeb%
        \bt@rangea=0\range\relax%
        \pgfmathtruncatemacro\result{ ( #1 >= \bt@rangea) && (#1 <= \bt@rangeb) }%
        \ifnum\result=1\relax%
            \breakforeach%
            \global\let\bt@inrange\@firstoftwo%
        \fi%
    }%
    \bt@inrange%
}
\newcommand\bt@getrangeb{%
    \@ifnextchar\relax%
        {\bt@rangeb=\bt@rangea}%
        {\@getrangeb}%
}
\def\@getrangeb-#1\relax{%
    \ifx\relax#1\relax%
        \bt@rangeb=100000%   \maxdimen is too large for pgfmath
    \else%
        \bt@rangeb=#1\relax%
    \fi%
}
%%%%%%%%%%%%%%%%%%%%%%%%%%%%%%%%%%%%%%%%%%%%%%%%%%%%%%%%%%%%%%%%%%%%%%%%%%%%%%
%
% \btLstHL<overlay spec>{range list}
%
% TODO BUG: \btLstHL commands can not yet be accumulated if more than one overlay spec match.
%
\newcommand<>{\btLstHL}[1]{%
\only#2{\btIfInRange{\value{lstnumber}}{#1}{\color{yellow}\def\lst@linebgrdcmd{\color@block}}{\def\lst@linebgrdcmd####1####2####3{}}}%
}%

\newcommand<>{\btLstHLG}[1]{%
\only#2{\btIfInRange{\value{lstnumber}}{#1}{\color{green}\def\lst@linebgrdcmd{\color@block}}{\def\lst@linebgrdcmd####1####2####3{}}}%
}%
\makeatother
%%%%%%%%%%%%%%%%%%%%%%%%%%%%%%%%%

\newcommand{\showTOC}{
    \begin{frame}[noframenumbering,plain]
        \frametitle{Вы находитесь здесь}
        \tableofcontents[currentsection]
    \end{frame}
}



\title[Language Memory Models]{Пара слов про модели памяти языков программирования}
\author[Александр Юрьевич]{Александр Филатов\\ filatovaur@gmail.com \\ \url{https://github.com/Svazars/lang-mem-models-intro}}

\date{}

\usetheme{CambridgeUS}

% tikz
\usepackage{pgf}
\usepackage{tikz}
\usepackage{tikz-qtree}
\usetikzlibrary{arrows, automata, fit, shapes, shapes.multipart, trees, positioning}

\usepackage{array}
\usepackage{cancel}
\usepackage{hyperref}
\usepackage[normalem]{ulem}

\begin{document}


% tikz common
\newcommand{\uedge}[2]{(#1) edge node {} (#2)}
\newcommand{\cedge}[3]{(#1) edge[#3, line width=1mm] node {} (#2)}
\newcommand{\cedgel}[3]{(#1) edge[#3, line width=0.5mm] node {} (#2)}
\newcommand{\nodesize}{1cm}

\begin{frame}
  \titlepage
\end{frame}

\section{Знакомство}

\begin{frame}{Bio: Александр Филатов}

Уже 8 лет как JVM-инженер\footnote{\tiny Иван Углянский, Один день из жизни JVM-инженера, \url{https://habr.com/ru/company/jugru/blog/719614/}}

\begin{itemize}
    \item 2015 - 2019, Excelsior JVM with AOT compilation\footnote{\tiny\url{https://habr.com/ru/company/jugru/blog/437180/}}
    \item 2019 - now, Huawei, Languages and Compilers lab\footnote{\tiny\url{http://rnew.tilda.ws/excelsiorathuawei}}
\end{itemize}

Специализация -- рантайм JVM

Узкая специализация -- сборщики мусора

\pause
Область интересов: 
\begin{itemize}
    \item многопоточность
    \item слабые модели памяти
    \item корректность многопоточных структур данных
    \item автоматическое управление памятью
\end{itemize}
\end{frame}


\begin{frame}{Мое персональное когнитивное искажение}
Я много страдал, отлаживая:
\begin{itemize}
    \item баги компилятора
    \item своего параллельного кода
    \item чужих реализаций многопоточных структур данных
\end{itemize}

Очень хочу поручить это дело бездушной машине. Потому питаю слабость к формализмам и математической нотации.

\end{frame}


\begin{frame}{Disclaimer}

Лекция называется "пара слов про ..."\ и будет идти 1.5 часа.

\pause
\begin{itemize}
    \item В последующих слайдах будет много упрощений и далеко не вся важная информация будет рассказана.

    \item Лекция носит ознакомительно-развлекательный характер.

    \item По ходу дела мы больше углубимся в теорию и разные философские вопросы. Практикам рекомендую посмотреть интересный доклад про биржи, котировки и их обработку на Java\footnote<2->{{\tiny Роман Елизаров "Миллионы котировок в секунду на чистой Java"\ \url{https://youtu.be/j3wFOmRmSeg}}}.
\end{itemize}

\pause
Вы узнаете:
\begin{itemize}
    \item Что такое модели памяти
    \begin{itemize}
        \item Процессорных архитектур
        \item Языков программирования
    \end{itemize}
    \item Зачем они нужны
    \item Как они выглядят в дикой природе 
\end{itemize}

\end{frame}


\begin{frame}{Блиц-опрос}

Пожалуйста, поднимите руку, если вы:
\begin{itemize}
    \pause
    \item Запускали \url{http://deadlockempire.github.io/}
    \pause
    \item Прошли обучалку
    \pause
    \item Прошли все уровни
    \pause
    \item Вам показалось слишком просто
\end{itemize}

\pause
Спасибо!
\end{frame}


\begin{frame}{План выступления}
\tableofcontents
\end{frame}




\section{Зачем писать многопоточный код?}
\showTOC


\begin{frame}{Зачем писать параллельные программы?}

Зачем в языке программирования давать средства для написания параллельных(parallel)/конкурентных(concurrent) программ?\footnote{Rob Pike, Concurrency Is Not Parallelism, https://go.dev/blog/waza-talk}

\pause

Возможные варианты ответа:
\begin{itemize}
 \item Так принято еще с 2000-х годов
 \item Современное оборудование фантастически многоядерное
 \item Необходимо строить сложные системы, ориентированные на огромное число пользователей
 \item А как иначе написать 
 \begin{itemize}
    \item сервер обработки входящий соединений?
    \item многоагентную симуляцию?
    \item игру-песочницу типа Factorio/Minecraft/RollercoasterTycoon/etc
 \end{itemize}
\end{itemize}

\pause

В подавляющем большинстве случаев, написание параллельного/распределенного/многопоточного кода -- это оптимизация.

\end{frame}


\begin{frame}[fragile]{Альтернативы}
\framesubtitle{Bash}

Задача: подсчитать число слов в текстовом файле\footnote{\tiny\url{https://github.com/Svazars/lang-mem-models-intro/tree/main/samples/bash}}.

\pause
Последовательное исполнение
\begin{lstlisting}
wc 50_000_000.txt

real    0m 4,900s
user    0m 4,212s
sys     0m 0,240s
\end{lstlisting}

\pause 
Параллельное исполнение (2 процесса)
\begin{lstlisting}
{ ( head -n 25000000 50_000_000.txt | wc) & }
{ ( tail -n 25000000 50_000_000.txt | wc) & }
wait

real    0m 2,323s
user    0m 4,576s
sys     0m 1,084s
\end{lstlisting}

\end{frame}

\begin{frame}[fragile]{Альтернативы}
\framesubtitle{make}

Задача: скомпилировать большой проект на языке C.

\begin{lstlisting}
make -j8
\end{lstlisting}

Аналогичные инструменты: 
\begin{itemize}
    \item ant/maven (Java)
    \item groovy DSL (Jenkins)
    \item sbt (Scala)
    \item ...
\end{itemize}

\end{frame}

\begin{frame}{Параллелизм уровня процессов}
\framesubtitle{Обсуждение}

\begin{itemize}
 \item ОС создана для того, чтобы быстро, эффективно и надежно управлять процессами
 \item Независимые шаги вычислений можно выполнять в разных процессах
 \item Write programs that do one thing and do it well\footnote{\url{https://en.wikipedia.org/wiki/Unix_philosophy}}
\end{itemize}

\pause
Исполнение нескольких потоков вычислений внутри одного процесса не требуется.

\pause 
Параллелизм уровня потоков не нужен.

\pause
Многопоточность не нужна.

\pause
В чем недостатки подхода с использованием только процессов?

\end{frame}


\begin{frame}{Параллелизм уровня процессов}
\framesubtitle{Недостатки}
Процесс -- это идентифицируемая абстракция совокупности взаимосвязанных системных ресурсов на основе отдельного и {\color<3>[rgb]{1,0,0}{независимого виртуального адресного пространства}}\footnote{https://ru.wikipedia.org/wiki/Процесс\_(информатика)}.

\pause

Межпроцессное взаимодействие бывает: %TODO insert бывает, сказать что бывает и хорошо
\begin{itemize}
  \item Долгое (latency)
  \item Дорогое (throughput)
  \item Не всегда кросс-платформенное (Windows/POSIX API vs. protobuf)\footnote<2->{https://stackoverflow.com/questions/60649/cross-platform-ipc}
  \item Не всегда надежное (разрыв соединения, смерть процесса)
  \item Не всегда безопасное
\end{itemize}

\pause

% TODO: давайте пожертвуем независимостью, вдруг будет лучше
\end{frame}


\begin{frame}{Параллелизм уровня потоков}

Взаимодействие потоков исполнения (threads) внутри общего адресного пространства.

\pause
У каждого потока есть

\begin{itemize}
    \item машинный стек
    \item идентификатор
    \item обработчик сигналов
    \item приоритет в планировщике задач
    \item ...
    \pause
    \item быстрый доступ к общей разделяемой памяти
\end{itemize}

\pause
Ремарка: целенаправленно избегаю альтернатив 
\begin{itemize}
 \item green threads/lightweight threads
 \item fibers/coroutines
 \item actors
 \item agents in distributed computing
\end{itemize}

\end{frame}


\begin{frame}{Параллелизм уровня потоков}
\framesubtitle{Применения}

\begin{itemize}
 \item Базы данных. Пользователь пишет SQL-запрос, но одна СУБД одновременно обрабатывает много запросов. 
 % Иллюзия однозадачности.
 \pause

 \item Сервер. Клиент шлет запросы, получает ответы, но одна система допускает одновременную обработку нескольких соединений.

\pause
 
 \item Сборщик мусора. Вы создали объект, попользовались им и забыли. Не задумываясь о том, что где-то там есть сборщик мусора, которые отследит ненужность объекта и переиспользует память под следующий объект.
 \pause
 
  \begin{itemize}
    \item Независимые задачи (parallel marking, parallel sweeping)
    \pause
 
    \item Одновременные задачи (concurrent marking, concurrent copying)
    \pause
 
    \item Работоспособность при наличии блокирующих вызовов (поток завис в \texttt{epoll\_wait}, а мусор всё равно был собран)
  \end{itemize}

\pause
 
 \item ...

\end{itemize}
 
\end{frame}


\begin{frame}{Параллелизм уровня потоков}
\framesubtitle{Выводы}

  Весьма хорошая и актуальная оптимизация.

  Способ добиться одновременного исполнения в рамках одного процесса.
  \begin{itemize}
    \item Всё виртуальное адресное пространство -- общий ресурс
    \item Все CPU -- общий ресурс
    \item Быстрая скорость обмена информацией
    \item Возможность совместно \only<1-3>{использовать}\only<4>{{\color{red} изменять}} данные
  \end{itemize}

  \pause
  Есть общий ресурс -- есть проблемы.

  Deadlock, livelock, starvation, priority inversion, lock convoy, thundering herd problem, ABA problem, use-after-free ...

  \pause
  Сегодня не об этом.

\end{frame}




 


\section{Компилятор хотел как лучше, а получилось...}
\showTOC

\begin{frame}[fragile]{Классические оптимизации однопоточного кода}
\framesubtitle{Inventing reads}

\begin{lstlisting}
static int a;
void foo_1() {
    int tmp;
    while (tmp = a) {
        do_something_with(tmp);
    }
}
\end{lstlisting}

\pause

Имеет ли право компилятор сэкономить регистры и переписать функцию следующим образом?

\begin{lstlisting}
void foo_2() {
    while (a) {
        do_something_with(a);
    }
}
\end{lstlisting}

\pause

Если в программе существуют другие потоки, изменяющие переменную \texttt{a} -- такое преобразование безопасно?
\end{frame}


\begin{frame}[fragile]{Классические оптимизации однопоточного кода}
\framesubtitle{Removing reads}

\begin{lstlisting}
static int a;
void foo_1() {
    int tmp;
    while (tmp = a) {
        do_something_with(tmp);
    }
}
\end{lstlisting}

\pause

Имеет ли право компилятор уменьшить количество загрузок из памяти и переписать функцию следующим образом?

\begin{lstlisting}
void foo_3() {
    int tmp = a;
    if (tmp)
        for (;;) do_something_with(tmp);
}
\end{lstlisting}

\pause

Если в программе существуют другие потоки, изменяющие переменную \texttt{a} -- такое преобразование безопасно?
\end{frame}


\begin{frame}[fragile]{Классические оптимизации однопоточного кода}
\framesubtitle{Godbolt}

\begin{adjustbox}{width=\textwidth,height=1cm,keepaspectratio}
\begin{lstlisting}
static int a;
void foo_1() {
    int tmp;
    while (tmp = a) {
        do_something_with(tmp);
    }
}
\end{lstlisting}
\end{adjustbox}

\texttt{x86-64 clang 16.0.0 -O2}\footnote{\url{https://godbolt.org/z/51W5adoTG}}:

\begin{adjustbox}{width=\textwidth,height=2cm,keepaspectratio}
\begin{lstlisting}
foo_1:                                  
        push    rbx
        mov     ebx, dword ptr [rip + a]
        test    ebx, ebx
        je      .LBB1_2
.LBB1_1:                           #<---------|    
        mov     edi, ebx           #          |
        call    do_something_with  #          |
        jmp     .LBB1_1            #----------|
.LBB1_2:
        pop     rbx
        ret
\end{lstlisting}
\end{adjustbox}

\end{frame}


\begin{frame}{Классические оптимизации однопоточного кода}
\framesubtitle{Конфликт интересов}

\begin{itemize}
    \item Хотим мощный оптимизирующий компилятор, чтобы наш однопоточный код работал как можно быстрее
    \item Не хотим, чтобы преобразования программ ломали наш многопоточный код, который улучшает производительность
\end{itemize}

\pause
Классический пример конфликтующих оптимизаций.

\pause
Как бы вы решали данную проблему?

\pause
\begin{itemize}
 \item Запретить какие-то преобразования (blacklist)
 \item Разрешить только "правильные"\ преобразования (whitelist)
\end{itemize}

\end{frame}

\begin{frame}{Вызовы для авторов языков}

Необходимо понять, какие преобразование "подозрительные"\ и включить нужные ремарки в спецификацию языка.

\pause
\begin{itemize}
 \item Как понять, что запреты исчерпывающие? 
 \item Как проверить, что указания непротиворечивы?
\end{itemize}

\pause
В примере "removing reads"\ используются довольно простые классические преобразования. Компиляторная классика -- CSE, DCE, UCE, loop-invariant code motion.

\pause
Надо написать некоторую спецификацию того, как различные потоки видят изменения разделяемой памяти (переменных, объектов, полей). 

\pause
Соблюсти баланс между \textit{понятностью}, \textit{производительностью} и \textit{устойчивостью}.

\end{frame}

\begin{frame}{Выводы}

\begin{itemize}
    \item Очевидные и верные преобразования однопоточных программ очень часто искажают поведение многопоточного кода, использующего общую память 
    \item Если язык программирования не готов радикально отказаться от общего изменяемого состояния\footnote{\tiny\url{https://en.wikipedia.org/wiki/Actor_model}}\textsuperscript{,}\footnote{\tiny\url{https://en.wikipedia.org/wiki/Communicating_sequential_processes}} или попытался, но не смог\footnote{\tiny\url{https://kotlinlang.org/docs/multiplatform-mobile-concurrency-overview.html#global-state}}, то необходимо определить "границы дозволенного"\ для оптимизаций и не очень сильно удивлять пользователей языка
    \item Language memory model -- описание того, какие есть гарантии у различных потоков приложения при обращении к разделяемым ячейкам памяти (переменным, объектам, полям, элементам массивов ...)
\end{itemize}

\end{frame}
 


\section{Процессор хотел как лучше, а получилось...}
\showTOC

\begin{frame}{Кругом враги}

Не только компиляторы (software) пытаются сломать ваше представление об исполнении программы в многопоточном контексте. 

\pause
Есть еще процессор и подсистема памяти (hardware). 

\pause
Которые умеют:
\begin{itemize}
    \pause
    \item Исполнять независимые инструкции одновременно (out-of-order execution)

    \pause
    \item Задействовать одни и те же ресурсы для исполнения логически независимых потоков (hyper-threading)

    \pause
    \item Спекулировать\footnote<6->{\tiny\url{https://en.wikipedia.org/wiki/Spectre_(security_vulnerability)}}\textsuperscript{,}\footnote<6->{\tiny\url{https://en.wikipedia.org/wiki/Meltdown_(security_vulnerability)}}
    \begin{itemize}
        \pause
        \item О предстоящих переходах (branch prediction)

        \pause
        \item О требуемой памяти (cache prefetching)

        \pause
        \item О результате вычислений (speculative execution)

        \pause
        \item И многом другом
    \end{itemize}
\end{itemize}

\end{frame}

\begin{frame}[fragile,t]{x86: Store buffering}


\begin{minted}[fontsize=\small]{c}
                          int x, y;
\end{minted}

\begin{tabular}{p{0.5\textwidth} p{0.5\textwidth}}

\begin{minted}[fontsize=\small]{c}
void threadA() {
      x = 1;
  int a = y;
}
\end{minted}

& 

\begin{minted}[fontsize=\small]{c}
void threadB() {                                   
        y = 1;                           
    int b = x;                           
}                    
\end{minted}
\end{tabular}


\pause



\begin{tabular}{p{0.5\textwidth} p{0.5\textwidth}}
\begin{minted}[fontsize=\small]{gas}
# thread A
MOV [x] ,  1  # (A.1)
MOV EAX , [y] # (A.2)
\end{minted}

& 

\begin{minted}[fontsize=\small]{gas}
# thread B          
MOV [y] , 1  # (B.1) 
MOV EBX, [x] # (B.2) 
\end{minted}
\end{tabular}

\end{frame}


\begin{frame}[fragile,t,noframenumbering]{x86: Store buffering}

\begin{tabular}{p{0.5\textwidth} p{0.5\textwidth}}
\begin{minted}[fontsize=\small]{gas}
# thread A
MOV [x] ,  1  # (A.1)
MOV EAX , [y] # (A.2)
\end{minted}

& 

\begin{minted}[fontsize=\small]{gas}
# thread B          
MOV [y] , 1  # (B.1) 
MOV EBX, [x] # (B.2) 
\end{minted}
\end{tabular}

\pause
Какие значения для \texttt{(EAX EBX)} допустимы?

\texttt{\ \ \ \ \ \ \ \ \ \ \ \ \ \ \ \ \ \ (1 1)\ , (0 1)\ , (1 0)\ , (0 0)}

\pause
Варианты исполнения:
\begin{itemize}
    \item \texttt{A.1 -> A.2 -> B.1 -> B.2}
    \item \texttt{\ \ \ \ \ \ \       B.1 -> A.2 -> B.2}
    \item \texttt{\ \ \ \ \ \ \ \ \ \ \ \ \ \              B.2 -> A.2}
    \item \texttt{B.1 -> A.1 -> A.2 -> B.2}
    \item \texttt{\ \ \ \ \ \ \ \ \ \ \ \ \ \              B.2 -> A.2}
    \item \texttt{\ \ \ \ \ \ \       B.2 -> A.1 -> A.2}
\end{itemize}
\end{frame}

\begin{frame}[fragile,t,noframenumbering]{x86: Store buffering}

\begin{tabular}{p{0.5\textwidth} p{0.5\textwidth}}
\begin{minted}[fontsize=\small]{gas}
# thread A
MOV [x] ,  1  # (A.1)
MOV EAX , [y] # (A.2)
\end{minted}

& 

\begin{minted}[fontsize=\small]{gas}
# thread B          
MOV [y] , 1  # (B.1) 
MOV EBX, [x] # (B.2) 
\end{minted}
\end{tabular}

Какие значения для \texttt{(EAX EBX)} допустимы?

\texttt{\ \ \ \ \ \ \ \ \ \ \ \ \ \ \ \ \ \ (1 1)\ , (0 1)\ , (1 0)\ , (0 0)}

Варианты исполнения:
\begin{itemize}
    \item \texttt{A.1 -> A.2 -> B.1 -> B.2}                            : \texttt{(0, 1)}
    \item \texttt{\ \ \ \ \ \ \       B.1 -> A.2 -> B.2}               : \texttt{(1, 1)}
    \item \texttt{\ \ \ \ \ \ \ \ \ \ \ \ \ \              B.2 -> A.2} : \texttt{(1, 1)}
    \item \texttt{B.1 -> A.1 -> A.2 -> B.2}                            : \texttt{(1, 1)}
    \item \texttt{\ \ \ \ \ \ \ \ \ \ \ \ \ \              B.2 -> A.2} : \texttt{(1, 1)}
    \item \texttt{\ \ \ \ \ \ \       B.2 -> A.1 -> A.2}               : \texttt{(1, 0)}
\end{itemize}
\end{frame}

\begin{frame}[fragile,t,noframenumbering]{x86: Store buffering}

\begin{tabular}{p{0.5\textwidth} p{0.5\textwidth}}
\begin{minted}[fontsize=\small]{gas}
# thread A
MOV [x] ,  1  # (A.1)
MOV EAX , [y] # (A.2)
\end{minted}

& 

\begin{minted}[fontsize=\small]{gas}
# thread B          
MOV [y] , 1  # (B.1) 
MOV EBX, [x] # (B.2) 
\end{minted}
\end{tabular}

Какие значения для \texttt{(EAX EBX)} допустимы?

\texttt{Ответ:\ \ \ \ \ \ \ \ \ \ \ \ (1 1)\ , (0 1)\ , (1 0)}

Варианты исполнения:
\begin{itemize}
    \item \texttt{A.1 -> A.2 -> B.1 -> B.2}                            : \texttt{(0, 1)}
    \item \texttt{\ \ \ \ \ \ \       B.1 -> A.2 -> B.2}               : \texttt{(1, 1)}
    \item \texttt{\ \ \ \ \ \ \ \ \ \ \ \ \ \              B.2 -> A.2} : \texttt{(1, 1)}
    \item \texttt{B.1 -> A.1 -> A.2 -> B.2}                            : \texttt{(1, 1)}
    \item \texttt{\ \ \ \ \ \ \ \ \ \ \ \ \ \              B.2 -> A.2} : \texttt{(1, 1)}
    \item \texttt{\ \ \ \ \ \ \       B.2 -> A.1 -> A.2}               : \texttt{(1, 0)}
\end{itemize}
\end{frame}

\begin{frame}[fragile,t,noframenumbering]{x86: Store buffering}

\begin{tabular}{p{0.5\textwidth} p{0.5\textwidth}}
\begin{minted}[fontsize=\small]{gas}
# thread A
MOV [x] ,  1  # (A.1)
MOV EAX , [y] # (A.2)
\end{minted}

& 

\begin{minted}[fontsize=\small]{gas}
# thread B          
MOV [y] , 1  # (B.1) 
MOV EBX, [x] # (B.2) 
\end{minted}
\end{tabular}

Какие значения для \texttt{(EAX EBX)} допустимы?

\only<1>{\texttt{Ответ:}}\only<2->{\texttt{Правильный ответ:}} \only<1>{\texttt{\ \ \ \ \ \ \ \ \ \ \ }} \texttt{(1 1)\ , (0 1)\ , (1 0)}
\pause
\texttt{, (0 0)}

\pause
Процессор может переупорядочить записи и чтения, если это не нарушает intra-thread order.
\pause
В данном случае изменился наблюдаемый другими процессорами порядок store и load операций\footnote<4->{\tiny\url{https://habr.com/ru/company/JetBrains-education/blog/523298/}}\textsuperscript{,}\footnote<4->{\tiny\url{https://diy.inria.fr/doc/SB.litmus}}\textsuperscript{,}\footnote<4->{\tiny\url{https://www.cl.cam.ac.uk/~pes20/weakmemory/cacm.pdf}}.

\pause
Вывод: порядок инструкций в машинном коде $\neq$ порядок наблюдаемых эффектов этих инструкций.

\end{frame}


\begin{frame}[fragile]{arm64: Independent Reads of Independent Writes}


\begin{tabular}{p{0.2\textwidth} p{0.2\textwidth} p{0.2\textwidth} p{0.2\textwidth}}
\begin{minted}[fontsize=\small]{python}
thread1
  x = 1
\end{minted}

& 

\begin{minted}[fontsize=\small]{python}
thread2
  y = 1
\end{minted}

&

\begin{minted}[fontsize=\small]{python}
thread3
 r1 = x
 r2 = y 
\end{minted}

&

\begin{minted}[fontsize=\small]{python}
thread4
 r3 = y
 r4 = x
\end{minted}
\end{tabular}


\pause
Может ли быть так, что \texttt{(r1 = 1, r2 = 0, r3 = 1, r4 = 0)}?

%\pause
%Могут ли потоки 3 и 4 увидеть изменения в разном порядке?

\pause
При условии, что переупорядочивание чтений не происходит.

\pause
\begin{itemize}
\item На \texttt{x86} или \texttt{x86\_64} (TSO): нет

\pause
\item На \texttt{ARM} или \texttt{POWER}: да\footnote<5->{\tiny A Tutorial Introduction to the ARM and POWER Relaxed Memory Models, section 6.1}
\end{itemize}

\pause
Записи могут "доехать"\ до других процессоров в разном порядке.

\pause
У каждого процессора своя временная шкала и некоторое видение окружающего мира. Возможно, отличающееся от других процессоров.

\pause
Вывод: нельзя рассматривать "запись в ячейку памяти"\ как точку на единой временной шкале\footnote<8->{\tiny The Art of Multiprocessor Programming by Maurice Herlihy \& Nir Shavit, Chapter 3 "Concurrent Objects"}.
\end{frame}


% \begin{frame}{Домашнее задание}
% 
% Постойте, постойте!
% 
% \pause
% На предыдущих слайдах говорится, что ячейки памяти пишутся и читаются разными процессорами без каких-либо разумных гарантий на порядок.
% 
% \pause
% А на курсе по архитектуре компьютера нам рассказывали, что есть специальные алгоритмы когерентности кэшей\footnote<3->{\tiny\url{https://en.wikipedia.org/wiki/Cache_coherence}}\textsuperscript{,}\footnote<3->{\tiny\url{https://en.wikipedia.org/wiki/MESI_protocol}}, которые не допускают рассинхронизации памяти.
% 
% \pause
% Кто-то из уважаемых преподавателей что-то недоговаривает.
% 
% \pause
% И человек около проектора пришел всего на одну лекцию, а потом исчезнет.
% 
% \pause
% Поразмыслите, как так выходит, что когерентность кэшей есть, а единого порядка видимых эффектов над различными ячейками памяти нет.
% \end{frame}


\begin{frame}{Почему так сложно?}

\begin{itemize}
 \item порядок инструкций в машинном коде $\neq$ порядок наблюдаемых эффектов этих инструкций

 \pause
 \item нельзя рассматривать "чтение/запись в ячейку памяти"\ как точку на единой временной шкале

 \pause
 \item у каждого процессора свои правила
\end{itemize}

\pause
Почему вообще хоть кто-то пользуется ARM/POWER/RISC-V и другими процессорами со слабой моделью памяти?

\pause
\begin{itemize}
  \item производительность
  \pause
  \item Производительность!
  \pause
  \item энергосбережение :)
\end{itemize}

\end{frame}

 


\section{Программист хочет как лучше...}
\showTOC

\begin{frame}[t]{Проблема №1}

Компилятор, в погоне за производительностью, начинает "фантазировать":
\begin{itemize}
	\pause
	\item Добавлять операции, которых не было в исходной программе
	\item Удалять написанные в исходном тексте операции
	\item Изменять порядок операций
\end{itemize}

\pause
Программисту хотелось бы \only<3>{явно сказать "делай в точности как написано"}\only<4->{иметь языковые средства, позволяющие контролировать происходящее}.
\pause

\end{frame}

\begin{frame}[t, fragile]{Compiler barriers}

\vspace{-0.5cm}
\begin{tabular}{p{0.5\textwidth} p{0.5\textwidth}}
\begin{minted}[fontsize=\small]{c}
int x, y;
void foo() {
      x = 1;
      y = 2;
      x = 3;
}
\end{minted}
&
\end{tabular}

\vspace{-0.5cm}

\pause

\begin{tabular}{p{0.5\textwidth} p{0.5\textwidth}}

\begin{minted}[fontsize=\small]{gas}
foo1:
    mov [y], 2
    mov [x], 3
    ret
\end{minted}

& 

\begin{minted}[fontsize=\small]{gas}
foo2:
    mov [x], 1
    mov [y], 2
    mov [x], 3
    ret
\end{minted}

\end{tabular}

\pause
Современный оптимизирующий компилятор выберет вариант слева\footnote<3->{\tiny\url{https://godbolt.org/z/r6sMzrj6K}}.

\pause
Но можно ему сказать -- "вот эта точка в программе -- барьер, она запрещает двигать операции через него".

\end{frame}


\begin{frame}[t, fragile]{Compiler barriers}

\vspace{-0.5cm}
\begin{tabular}{p{0.5\textwidth} p{0.5\textwidth}}
\begin{minted}[fontsize=\small]{c}
int x, y;
void foo1() {
      x = 1;
      y = 2;
      x = 3;
}
\end{minted}

& 

\begin{minted}[fontsize=\small]{c}
int x, y;
void foo2() {
      x = 1;
      barrier();
      y = 2;
      x = 3;
}
\end{minted}
\end{tabular}

\pause
\vspace{-0.5cm}

\begin{tabular}{p{0.5\textwidth} p{0.5\textwidth}}

\begin{minted}[fontsize=\small]{gas}
foo1:
    mov [y], 2
    mov [x], 3
    ret
\end{minted}

& 

\begin{minted}[fontsize=\small]{gas}
foo2:
    mov [x], 1
    mov [y], 2
    mov [x], 3
    ret
\end{minted}
\end{tabular}

\pause
Обратите внимание, что это ОЧЕНЬ низкоуровневый механизм и при написании современного C/C++ кода его использовать не следует\footnote<3->{\tiny\url{https://preshing.com/20120625/memory-ordering-at-compile-time/}}.

\end{frame}


\begin{frame}{Проблема №2}
Процессор, в погоне за производительностью, начинает "чудить":
\begin{itemize}
	\pause
	\item С разной скоростью передавать информацию другим процессорам 
	\item Изменять порядок операций
\end{itemize}

\pause
Программисту хотелось бы иметь языковые средства, позволяющие контролировать происходящее.

\pause
В многоядерном процессоре предусмотрены специальные инструкции, которые позволяют установить порядок "видимости"\ среди операций чтения и записи\footnote<4->{\tiny\url{https://en.wikipedia.org/wiki/Memory_barrier}}.

\end{frame}

\begin{frame}[fragile]{Memory barriers}
\framesubtitle{Пример}

\vspace{-0.5cm}

\begin{tabular}{p{0.5\textwidth} p{0.5\textwidth}}
\begin{minted}[fontsize=\small]{gas}
# thread A
mov [x] ,  1
mov EAX , [y]
\end{minted}

& 

\begin{minted}[fontsize=\small]{gas}
# thread B          
mov [y] , 1
mov EBX, [x]
\end{minted}
\end{tabular}

Допустимые результаты: 
\pause 
\texttt{(0, 0) (1, 0) (0, 1) (1, 1)}

\pause
\vspace{-0.25cm}

\begin{tabular}{p{0.5\textwidth} p{0.5\textwidth}}
\begin{minted}[fontsize=\small]{gas}
# thread A
mov [x] ,  1
mfence      
mov EAX , [y]
\end{minted}

& 

\begin{minted}[fontsize=\small]{gas}
# thread B          
mov [y] , 1
mfence
mov EBX, [x]
\end{minted}
\end{tabular}

\pause
Инструкция \texttt{mfence} отменяет переупорядочивание операций с памятью на уровне процессора\footnote<4->{\tiny\url{https://www.felixcloutier.com/x86/mfence.html}}.
-
\pause

Допустимые результаты: \pause \texttt{(1, 0) (0, 1) (1, 1)}
\end{frame}


\begin{frame}{Memory barriers}
\framesubtitle{Предостережение}

Не забывайте
\begin{itemize}
  \pause
  \item Барьеры дорогие

  \pause
  \item Инструкции обладают процессорно-специфичной и иногда весьма запутанной семантикой

  \pause
  \item Расстановка барьеров в ряде многопоточных алгоритмов -- не единственная и, по сути, является смесью искусства и инженерного мастерства

  \pause
  \item Практически все, кто так делает, похожи на глотателей огня. Даже если живы, то со шрамами от ожогов\footnote<5->{\tiny\url{https://www.researchgate.net/publication/228824849_Memory_Barriers_a_Hardware_View_for_Software_Hackers}}.
\end{itemize}
\end{frame}


\begin{frame}[t]{Языковые средства}
\framesubtitle{Вопросы взаимного влияния}

Краткая выжимка из предыдущих слайдов. Для "починки"\ многопоточных алгоритмов бывает необходимо:
\begin{itemize}
  \item запрещать компилятору "фантазировать"
  \item запрещать процессору "чудить"
\end{itemize}

\pause
Например, с помощью барьеров разного вида. 

\pause
Кажется, что компиляторные и процессорные барьеры -- это разные сущности и, наверное, независимые друг от друга. 

\pause
Должен ли компилятор уважать написанные программистом в исходном коде процессорные барьеры?

\pause
Должен ли процессор уважать написанные программистом в исходном коде компиляторные барьеры?
\end{frame}

\begin{frame}[t, noframenumbering]{Языковые средства}
\framesubtitle{Вопросы взаимного влияния}

Краткая выжимка из предыдущих слайдов. Для "починки"\ многопоточных алгоритмов бывает необходимо:
\begin{itemize}
  \item запрещать компилятору "фантазировать"
  \item запрещать процессору "чудить"
\end{itemize}

Нужен ли программисту независимый контроль над обеими проблемами, с учетом того, что допустить ошибку невероятно легко, а негативные последствия практически невозможно отладить?

\end{frame}


\begin{frame}[t]{Языковые средства}
\framesubtitle{Два мира}

Краткая выжимка из предыдущих слайдов. Для "починки"\ многопоточных алгоритмов бывает необходимо:
\begin{itemize}
  \item запрещать компилятору "фантазировать"
  \item запрещать процессору "чудить"
\end{itemize}

Существуют разные подходы:
\begin{itemize}
  
  \pause
  \item Языки, с маниакальной страстью пытающиеся дать разработчикам способы написать очень быструю, но некорректную программу.
  \pause
  C/C++

  \pause
  \item Языки, сфокусированные на безопасности и с помощью управляемой среды создающие "песочницу".
  \pause
  Java
\end{itemize}
\end{frame}


\begin{frame}[t,noframenumbering]{Языковые средства}
\framesubtitle{Два мира}

Краткая выжимка из предыдущих слайдов. Для "починки"\ многопоточных алгоритмов бывает необходимо:
\begin{itemize}
  \item запрещать компилятору "фантазировать"
  \item запрещать процессору "чудить"
\end{itemize}

Существуют разные подходы:
\begin{itemize}
  
  \item Языки,\only<1-2>{ с маниакальной страстью пытающиеся дать разработчикам способы написать очень быструю, но некорректную программу}\only<3>{ \sout{с маниакальной страстью пытающиеся дать разработчикам способы написать очень быструю, но некорректную программу}}\only<4->{ {\color{red}предоставляющие разработчикам документированное и низкоуровневое API для написания очень быстрых программ}}.
  C/C++

  \item Языки, сфокусированные на безопасности и с помощью управляемой среды\only<1>{ \sout{создающие}}\only<2->{ {\color{red}пытающиеся создать}} "песочницу".
  Java
\end{itemize}

\end{frame}


\begin{frame}[fragile, t]{Разные миры}
\framesubtitle{Пример}

\begin{minted}[fontsize=\small]{c}
static volatile int x, y;
\end{minted}

\begin{tabular}{p{0.5\textwidth} p{0.5\textwidth}}

\begin{minted}[fontsize=\small]{c}
void threadA() {
      x = 1;
  int a = y;
}
\end{minted}

& 

\begin{minted}[fontsize=\small]{c}
void threadB() {                                   
        y = 1;                           
    int b = x;                           
}                    
\end{minted}
\end{tabular}

\pause
\vspace{-0.5cm}
C\footnote<2->{\tiny\url{https://godbolt.org/z/q4raxqrTe}}:

\begin{tabular}{p{0.5\textwidth} p{0.5\textwidth}}
\begin{minted}[fontsize=\small]{gas}
mov [x] ,  1
mov EAX , [y]
\end{minted}

& 

\begin{minted}[fontsize=\small]{gas}
mov [y] , 1
mov EBX, [x]
\end{minted}
\end{tabular}

\pause
Допустимые результаты: \texttt{(0, 0) (1, 0) (0, 1) (1, 1)}
\end{frame}

\begin{frame}[fragile, t, noframenumbering]{Разные миры}
\framesubtitle{Пример}

\begin{minted}[fontsize=\small]{c}
static volatile int x, y; 
\end{minted}

\begin{tabular}{p{0.5\textwidth} p{0.5\textwidth}}

\begin{minted}[fontsize=\small]{c}
void threadA() {
      x = 1;
  int a = y;
}
\end{minted}

& 

\begin{minted}[fontsize=\small]{c}
void threadB() {                                   
        y = 1;                           
    int b = x;                           
}                    
\end{minted}
\end{tabular}

\vspace{-0.5cm}
Java\footnote{\tiny\url{https://github.com/Svazars/lang-mem-models-intro/tree/main/samples/java}}:


\begin{tabular}{p{0.5\textwidth} p{0.5\textwidth}}
\begin{minted}[fontsize=\small]{gas}
mov [x] ,  1
lock add [rsp], 0x0
mov EAX , [y]
\end{minted}

& 

\begin{minted}[fontsize=\small]{gas}
mov [y] , 1
lock add [rsp], 0x0
mov EBX, [x]
\end{minted}
\end{tabular}

\pause
\vspace{-0.5cm}
\texttt{lock add [rsp], 0x0} $\approx$ \texttt{mfence} $\approx$ full memory barrier\footnote<2->{\tiny\url{https://shipilev.net/blog/2014/on-the-fence-with-dependencies/}}

\pause
Допустимые результаты: \texttt{(1, 0) (0, 1) (1, 1)}

\end{frame}


\begin{frame}[fragile, t, noframenumbering]{Разные миры}
\framesubtitle{Пример}

\begin{minted}[fontsize=\small]{c}
static volatile int x, y; 
\end{minted}

\begin{tabular}{p{0.5\textwidth} p{0.5\textwidth}}
\begin{minted}[fontsize=\small]{c}
void threadA() {
      x = 1;
  int a = y;
}
\end{minted}

& 

\begin{minted}[fontsize=\small]{c}
void threadB() {                                   
        y = 1;                           
    int b = x;                           
}                    
\end{minted}
\end{tabular}

Допустимые результаты в C: \texttt{(0, 0) (1, 0) (0, 1) (1, 1)}

Допустимые результаты в Java: \texttt{(1, 0) (0, 1) (1, 1)}

\pause
Обычно пишут, что \texttt{volatile} в языке Си ограничивает только преобразования на уровне компилятора, а в языке Java -- ограничивает как компилятор, так процессор.

\pause
Но давать слишком простую модель реального мира -- плохо с педагогической точки зрения.

\pause
Поэтому защищу себя от гнева ваших будущих работодателей.

\end{frame}


\begin{frame}{Совет от мудрой совы}

В языке С вместо \texttt{volatile} \textbf{всегда} используйте типы из \texttt{stdatomic.h}\footnote{\tiny\url{https://en.cppreference.com/w/c/language/atomic}} для разделяемых переменных.

\pause

Если стандарт C11 не поддерживается используемым компилятором или в коде обычные переменные изменяются из разных потоков одновременно -- бегите\footnote<2->{\tiny\url{https://www.kernel.org/doc/Documentation/process/volatile-considered-harmful.rst}}.

\pause

Для любителей почитать как ругается Линус Торвальдс, советую обратить внимание на цепочку обсуждений "volatile considered evil"\footnote<3->{\tiny\url{https://lkml.org/lkml/2006/7/6/159}}:

\begin{quote}
 "volatile"\ really \_is\_ misdesigned. The semantics of it are so unclear as to be totally useless. The only thing "volatile"\ can ever do is generate worse code, WITH NO UPSIDES.
\end{quote}

\end{frame}


\begin{frame}[fragile]{Языковые средства написания работающих многопоточных программ}
\framesubtitle{Выводы}

\begin{itemize}

 \pause
 \item Пожалуйста, никогда и никому не говорите что \texttt{volatile} в С и в Java имеют один и тот же смысл
 
 \pause
 \item Не думайте, что бездумное добавление \textt{volatile} в вашу программу сделает её корректной (даже на Java это не так)
 
 \pause
 \item Всегда помните, что в разных языках "одинаковая"\ конструкция может иметь весьма разный смысл
 \begin{itemize}
    \item смена языкового стека
    \item адаптация алгоритма из библиотеки/публикации
    \item кросс-языковая трансляция
 \end{itemize}

 \pause
 \item Компиляторные и процессорные барьеры -- прямолинейный способ добиться желаемого при написании многопоточных программ

 \pause
 \item Барьерный подход несет с собой много неявной сложности и специфики (языка, компилятора, процессора)
\end{itemize}

\end{frame}

 

\section{Language memory models. Примеры из жизни.}
\showTOC

\begin{frame}[t]{Есть ли более простые решения?}

Я простой Java-программист, я не хочу даже думать о тысячах оптимизаций, которыми компилятор может сломать мою программу.

\pause
Также я не хочу читать тысячи страниц спецификации процессорной архитектуры, чтобы расставлять какие-то барьеры. 

\pause
\only<3-4>{
Intel® 64 and IA-32 Architectures Software Developer’s Manual:
\begin{itemize}
	\item Volume 1: Basic Architecture (458 страниц)
	\item Volume 2: Instruction Set Reference, A-Z (1513 страниц)
	\item Volume 3: System Programming Guide (1638 страниц)
\end{itemize}
}

\only<4>{Размеры мануала по ARM64 (v8, для конкретики) сами найдите.}

\pause
\pause
Хочу кросс-платформенный код писать и чтобы он был понятный, простой и поддерживаемый.

\pause

Требуется описание операций с памятью и их свойств в используемом языке программирования
\end{frame}


\begin{frame}[t]{Есть ли более простые решения?}

Я простой Java-программист, я не хочу даже думать о тысячах оптимизаций, которыми компилятор может сломать мою программу.

Также я не хочу читать тысячи страниц спецификации процессорной архитектуры, чтобы расставлять какие-то барьеры. 

Хочу кросс-платформенный код писать и чтобы он был понятный, простой и поддерживаемый.

Требуется описание операций с памятью и их свойств в используемом языке программирования
\begin{itemize}
	\item Независимое от платформы (ОС, процессорная архитектура) 
	\item Независимое от среды исполнения (компилятор, сборщик мусора)
	\item Независимое от версии языка\footnote{Недостижимый идеал или суровая действительность обратной совместимости?} 
\end{itemize}

\pause
Требуется language memory model.
\end{frame}


\begin{frame}{Language Memory Model}

\begin{itemize}
    \item  Как писать такой документ? Литературным английским? В виде алгоритма? В виде набора разрешающих правил? В виде набора запрещающих правил? 

    \pause
    \item  Должна ли спецификация меняться от версии к версии? 

    \pause
    \item  Лучше чтобы она была более строгой или более слабой? 

    \pause
    \item  Можно ли проверить что придуманные правила согласованы между собой?   

    \pause
    \item  Можно ли гарантировать, что любая программа будет адекватно и однозначно описываться придуманной моделью?

    \pause
    \item  Существует ли какой-то специальный математический аппарат, облегчающий написание спецификации?

    \pause
    \item  А пользователи языка должны смочь это прочитать? Понять? Применить на практике?
\end{itemize}

\pause
Соблюсти баланс между \textit{понятностью}, \textit{производительностью} и \textit{устойчивостью}
\end{frame}

\begin{frame}[t]{О практической пользе данной лекции}

\only<1>{
На слайде~\insertframenumber~самое время задаться именно таким вопросом.
}

\only<2-3>{
С точки зрения большинства прикладных программистов, модель памяти не нужна.
}
\only<4->{
\sout{С точки зрения большинства прикладных программистов, модель памяти не нужна.}
}

\only<3>{
С другой стороны, большинство программистов, согласно опросам из интернета, пишет на таких языках как Python, JavaScript, VisualBasic, PHP.
}

\only<5->{
С точки зрения большинства прикладных программистов, модель памяти не должна мешать <<делать дело>>.
}

\only<6->{
А если случается необъяснимая бесовщина, то Senior Software Engineer <<придёт и молча поправит всё>>.
}

\only<7->{
Глупо скрывать от вас тот факт, что материал про модели памяти -- узкоспециализированный. 
}

\only<8->{С другой стороны, вы уже не доверяете компилятору и процессору.} \only<9->{Осталось совсем немного.}

\only<10>{Потеряйте веру в людей ;)}\only<11->{Потеряйте веру в \sout{людей} дизайнеров языков программирования.}

\end{frame}

\begin{frame}{Holy war warning}

Следующие несколько слайдов могут бросить тень на ваш любимый язык программирования.

\pause

Каждый из упомянутых промышленных языков программирования имеет спецификацию и, в том числе, описание модели памяти. Я педантично приведу соответствующие ссылки.
Но продолжу говорить, что у некоторых языков нет\only<3->{\ {\color{red} вменяемой}} модели памяти.

\pause
\pause

Порядок упоминания языков не соответствует <<качеству>> языка, просто так мне проще выстроить повествование.

\pause

Поехали.

\end{frame}


\begin{frame}[fragile, t]{Существующие подходы к описанию моделей памяти}
\framesubtitle{Нет человека -- нет проблемы}

Если в программе нет data race -- то нет необходимости говорить о модели памяти для многопоточных сред.

\pause

Поэтому можно использовать \textit{правильные} подходы к программированию:
\begin{itemize}
	\pause
	\item Неизменяемые структуры данных

	\pause
	\item Декларативное описание вычислений
\end{itemize}

\pause
Clojure

\pause
Haskell

\pause
\begin{quote}
All told, a monad in X is just a monoid in the category of endofunctors of X, with product × replaced by composition of endofunctors and unit set by the identity endofunctor.
\end{quote}

\pause
Дело в языке или в технике программирования?

\end{frame}


\begin{frame}[fragile, t]{Существующие подходы к описанию моделей памяти}
\framesubtitle{Immutability}

Если структура данных неизменяема -- то 

\begin{itemize}
	\pause
	\item разные потоки могут одновременно наблюдать её (читать из разделяемых ячеек памяти)

	\pause
	\item невозможны конфликтующие операции	
\end{itemize}

\pause
Невероятно удобно!
\pause
Но есть нюанс...

\pause
А как обновлять такую структуру данных в связи с изменением внешних условий?

\pause
Создать новый неизменяемый экземпляр, который содержит самую актуальную информацию.

\pause
Возникают определенные трудности:
\begin{itemize}
	\pause
	\item Издержки на пересоздание крупных структур

	\pause
	\item Операция публикации -- это операция записи, не так ли? 
\end{itemize}
\end{frame}

\begin{frame}[fragile, t]{Существующие подходы к описанию моделей памяти}
\framesubtitle{Declarative DSL}

Описать требуемый результат, а система сама разберется, как воспользоваться параллелизмом при организации вычислений:

\pause
\begin{itemize}
	\item OpenMP {\tiny\url{https://www.openmp.org/}}
	\item Intel TBB {\tiny\url{https://github.com/oneapi-src/oneTBB}}
	\item MPI {\tiny\url{https://www.open-mpi.org/}}
	\item Java parallel streams {\tiny\url{https://docs.oracle.com/javase/tutorial/collections/streams/parallelism.html}}
	\item MapReduce {\tiny\url{https://research.google/pubs/pub62/}}
	\item Resilient Distributed Datasets {\tiny\url{https://dl.acm.org/doi/10.5555/2228298.2228301}}
\end{itemize}

\pause
Абстракции текут\footnote<3->{\tiny\url{https://www.joelonsoftware.com/2002/11/11/the-law-of-leaky-abstractions/}} и в большинстве случаев вносят издержки.

\pause
При реализации таких DSL неизбежно нужно записывать значения в разделяемые ячейки памяти, не так ли?
\end{frame}


\begin{frame}{Существующие подходы к описанию моделей памяти}
\framesubtitle{Альтернативные подходы}

Модель программирования без (явного) разделяемого состояния.
\pause
\begin{itemize}
	\item Communicating sequential processes
	\item Actor model
\end{itemize}

\pause
Все вычислительные агенты (легковесные процессы) независимы друг от друга и обмениваются неизменяемыми сообщениями.

\pause
\begin{itemize}
	\item Erlang\footnote<4->{\tiny\url{https://www.erlang.org/}}
	\item Akka actors\footnote<4->{\tiny\url{https://doc.akka.io/docs/akka/current/typed/actors.html#akka-actors}}
\end{itemize}

\pause
Код реализации таких систем (Erlang VM, Akka internals) использует разделяемые ячейки памяти, чтобы передавать информацию от одного агента к другому, не так ли?
\end{frame}


\begin{frame}[fragile, t]{Существующие подходы к описанию моделей памяти}
\framesubtitle{Наивные подходы: запретить и не пущать}
Swift\footnote{\tiny\url{https://github.com/apple/swift-evolution/blob/main/proposals/0282-atomics.md}}

\only<1> {
\begin{quote}
Concurrent write/write or read/write access to the same location in memory generally remains undefined/illegal behavior, unless all such access is done through a special set of primitive atomic operations.
\end{quote}
}

\only<2> {
\begin{quote}
Concurrent write/write \sout{or read/write} access to the same location \sout{in memory generally} remains \sout{undefined/}illegal behavior, unless \sout{all such access} is done through \sout{a special set of primitive} atomic operations.
\end{quote}	
}

\only<3->{
\begin{quote}
Concurrent write/write access to the same location remains illegal behavior, unless is done through atomic operations.
\end{quote}
}

\begin{onlyenv}<4->
%\begin{lstlisting}
\begin{minted}[fontsize=\small]{swift}
import Foundation
class Bird {}
var S = Bird()
let q = DispatchQueue.global(qos: .default)
q.async { while(true) { S = Bird() } }
while(true) { S = Bird() }
\end{minted}
%\end{lstlisting}
\end{onlyenv}

\only<5->{
При запуске происходит ошибка \texttt{double free or corruption}.
}

\only<6->{
Почему? Попробуйте догадаться сами\footnote{\tiny\url{https://tonygoold.github.io/arcempire/}} или подсмотрите в решебник\footnote{\tiny\url{https://github.com/apple/swift/blob/main/docs/proposals/Concurrency.rst}}.
}

\end{frame}


\begin{frame}[t]{Существующие подходы к описанию моделей памяти}
\framesubtitle{Наивные подходы: reference implementation is a specification}

{\small
ECMA-334, \texttt{C\# language specification}\footnote{\tiny\url{https://www.ecma-international.org/publications-and-standards/standards/ecma-334}}: \texttt{14.5.4 Volatile fields} (2)
}

{\small
ECMA-335, \textt{CLI}\footnote{\tiny\url{https://www.ecma-international.org/publications-and-standards/standards/ecma-335/}}: \texttt{I.12.6 Memory model and optimizations} (4)
}
\end{frame}


\begin{frame}[t,noframenumbering]{Существующие подходы к описанию моделей памяти}
\framesubtitle{Наивные подходы: reference implementation is a specification}

{\small
ECMA-334, \texttt{C\# language specification}: \texttt{14.5.4 Volatile fields} (2)
}

{\small
ECMA-335, \textt{CLI}: \texttt{I.12.6 Memory model and optimizations} (4)
}

Модель памяти также описана для .NET\footnote{\tiny\url{https://github.com/dotnet/runtime/blob/main/docs/design/specs/Memory-model.md}}.
\pause
Официально несовместим с предыдущими документами.
\only<3>{
\begin{quote}
.NET runtime assumes that the side-effects of memory reads and writes include only observing and changing values at specified memory locations. This applies to all reads and writes - volatile or not. \textbf{This is different from ECMA model.}
\end{quote}
}

\end{frame}

\begin{frame}[t,noframenumbering]{Существующие подходы к описанию моделей памяти}
\framesubtitle{Наивные подходы: reference implementation is a specification}

{\small
ECMA-334, \texttt{C\# language specification}: \texttt{14.5.4 Volatile fields} (2)
}

{\small
ECMA-335, \textt{CLI}: \texttt{I.12.6 Memory model and optimizations} (4)
}

Модель памяти также описана для .NET, официально несовместима с предыдущими документами.

\pause
Модель памяти также описана для Mono\footnote{\tiny\url{https://www.mono-project.com/docs/advanced/runtime/docs/atomics-memory-model/}}, который старается сохранять конформность с .NET.
\only<3>{
\begin{quote}
... here is a quirk in the .NET implementation where these methods actually use the MemoryBarrier method to insert a barrier. This is stronger than a simple acquire or release barrier. We do the same for compatibility.
\end{quote}
}
\end{frame}


\begin{frame}[t,noframenumbering]{Существующие подходы к описанию моделей памяти}
\framesubtitle{Наивные подходы: reference implementation is a specification}

{\small
ECMA-334, \texttt{C\# language specification}: \texttt{14.5.4 Volatile fields} (2)
}

{\small
ECMA-335, \textt{CLI}: \texttt{I.12.6 Memory model and optimizations} (4)
}

Модель памяти также описана для .NET, официально несовместима с предыдущими документами.

Модель памяти также описана для Mono, который старается сохранять конформность с .NET.

\pause
Общий подход спецификации:
\begin{itemize}
	\pause
	\item \only<3>{Сказать, что соответствующие операции имеют 
	\begin{itemize}
		\item release semantics
		\item acquire semantics
		\item full-fence semantics
	\end{itemize}
	что бы это ни значило :)
	}
	\pause
	Использовать термины, не давая строгие определения.

	\pause
	\item Явно запретить некоторые перестановки операций с памятью.
\end{itemize}

\pause
Насколько полон список "запретных"\ оптимизаций, как будет система эволюционировать в будущем, будут ли еще отступления от стандарта не раскрыто.
\end{frame}

\begin{frame}[fragile, t]{Существующие подходы к описанию моделей памяти}
\framesubtitle{Наивные подходы: strict consistency}

Наиболее интуитивное определение для операций с памятью -- они все происходят атомарно и их все можно расположить на единой шкале времени\footnote{\tiny\url{https://en.wikipedia.org/wiki/Consistency_model#Strict_consistency}}.

\end{frame}


\begin{frame}[fragile, t, noframenumbering]{Существующие подходы к описанию моделей памяти}
\framesubtitle{Наивные подходы: strict consistency}

Наиболее интуитивное определение для операций с памятью -- они все происходят атомарно и их все можно расположить на единой шкале времени.


\begin{lstlisting}

void thread1() {      |    void thread2() {                                   
                      |
      foo()           |          baz()                           
                      |
                      |
      bar()           |          foo()                           
                      |
}                     |    }                    
\end{lstlisting}
\end{frame}

\begin{frame}[fragile, t, noframenumbering]{Существующие подходы к описанию моделей памяти}
\framesubtitle{Наивные подходы: strict consistency}

Наиболее интуитивное определение для операций с памятью -- они все происходят атомарно и их все можно расположить на единой шкале времени.

\begin{lstlisting}

void thread1() {      |    void thread2() {                                   
       lock()         |           lock()
      foo()           |          baz()                           
       unlock()       |           unlock()
       lock()         |           lock()
      bar()           |          foo()                           
       unlock()       |           unlock()
}                     |    }                    
\end{lstlisting}	

\end{frame}

\begin{frame}[fragile, t, noframenumbering]{Существующие подходы к описанию моделей памяти}
\framesubtitle{Наивные подходы: strict consistency}

Наиболее интуитивное определение для операций с памятью -- они все происходят атомарно и их все можно расположить на единой шкале времени.
Защищать глобальным мьютексом каждую операцию.

\begin{lstlisting}
static GlobalInterpreterLock GIL = ...;
void thread1() {      |    void thread2() {                                   
   GIL.lock()         |       GIL.lock()
      foo()           |          baz()                           
   GIL.unlock()       |       GIL.unlock()
   GIL.lock()         |       GIL.lock()
      bar()           |          foo()                           
   GIL.unlock()       |       GIL.unlock()
}                     |    }                    
\end{lstlisting}	

\end{frame}

\begin{frame}[fragile, t, noframenumbering]{Существующие подходы к описанию моделей памяти}
\framesubtitle{Наивные подходы: strict consistency}

Наиболее интуитивное определение для операций с памятью -- они все происходят атомарно и их все можно расположить на единой шкале времени.
Защищать глобальным мьютексом каждую операцию.
 
\pause
Прям как в Python!

\pause
Потоки в языке есть\footnote<3->{\tiny\url{https://docs.python.org/3/library/threading.html}}, просто их неэффективность является "особенностью"\ интерпретатора CPython.

\pause
Но PyPy тоже не собирается отказываться от GIL\footnote<4->{\tiny\url{https://doc.pypy.org/en/latest/faq.html#does-pypy-have-a-gil-why}}.

\pause
Попытка переделать модель языка пока не увенчалась успехом\footnote<5->{\tiny\url{https://peps.python.org/pep-0583/}}.

\pause
Просто так выбросить GIL мешает нежелание замедлять скриптовый язык еще на 10-20-30\%, ломая интероп с нативными библиотеками\footnote<6->{\tiny\url{https://peps.python.org/pep-0703/}}.
\end{frame}


\begin{frame}[fragile, t]{Существующие подходы к описанию моделей памяти}
\framesubtitle{Наивные подходы: strict consistency}

Наиболее интуитивное определение для операций с памятью -- они все происходят атомарно и их все можно расположить на единой шкале времени.

\pause
Пусть в языке вообще не будет потоков.

\pause
Один поток обрабатывает события, каждое из которых может породить другие, возможно отложенные, события.
\pause
Event loop.

\pause
Прям как в JavaScript!

\pause
Оказалось, что пользователи любят использовать все ядра своих систем.
\pause
Можно запускать дополнительных независимых агентов (web workers) и общаться сообщениями\footnote<7->{\tiny\url{https://www.w3schools.com/html/html5_webworkers.asp}}.
\pause
Можно разделять между агентами массивы байтов\footnote<8->{\tiny\url{https://developer.mozilla.org/en-US/docs/Web/JavaScript/Reference/Global_Objects/SharedArrayBuffer}}.
\pause
Получить data race и думать, что это значит\footnote<9->{\tiny "Repairing and Mechanising the JavaScript Relaxed Memory Model"\ \url{https://arxiv.org/abs/2005.10554}}.

\end{frame}


\begin{frame}[t]{Существующие подходы к описанию моделей памяти}
\framesubtitle{Наивные подходы: метод страуса}

C/С++ до стандартов C++11\footnote{\tiny\url{https://en.wikipedia.org/wiki/C\%2B\%2B11#Multithreading_memory_model}}

\pause
\begin{itemize}
	\item С точки зрения стандарта, потоков не существовало. Были библиотеки их реализующие.
	\item Каждый проект изобретал свои костыли. Специфичные компилятору, процессору, среде запуска. 
	\item Любой data race -- undefined behaviour. Не подходит для безопасных языков.
\end{itemize}

\pause
Экономия времени дизайнеров языка ценой увеличенных издержек прикладных программистов.
\end{frame}


\begin{frame}[t]{Существующие подходы к описанию моделей памяти}
\framesubtitle{Прагматичные подходы: дай человеку удочку}

C/С++ до соответствующих стандартов + POSIX threads\footnote{\tiny\url{https://en.wikipedia.org/wiki/Pthreads}}
%
\pause
\begin{itemize}
	\item С точки зрения стандарта, потоков не существовало. Появилась универсальная библиотека их реализующая.
	\item Костыли собраны в одном месте, исходнике библиотеки.
	\item Любой data race -- всё еще undefined behaviour. Use mutexes, Luke!
\end{itemize}

\pause
Во-первых, написание такой библиотеки представляет большой труд с кучей платформенно-специфичных трудностей.

\pause

Во-вторых, "Threads Cannot Be Implemented As a Library"\footnote<4->{\tiny\url{https://www.hpl.hp.com/techreports/2004/HPL-2004-209.pdf}}.
\pause

\begin{quote}
 The Pthreads specification prohibits races, i.e. accesses to a shared variable while another thread is modifying it. ... the
 problem here is that whether or not a race exists depends on the semantics of the programming language, which in turn requires that we have a properly defined memory model.
 Thus this definition is circular.
\end{quote}
%
%%Даже эксперты ошибаются TODO-link-for-libc-semaphore-probelm, link-for-libc-wait-notify-bug, link-to-membars-in-kernel-om-arm32, whatever else
%%И пользователи ворчат на тему недостаточной производительности, пишут свои (зачастую ненадежные) решения.
\end{frame}

\begin{frame}[t]{Существующие подходы к описанию моделей памяти}
\framesubtitle{Прагматичные подходы: дай человеку спиннинг}

C/С++ до 11 версии + POSIX threads + санитайзеры

\pause

Динамические проверки свойств программы:
\begin{itemize}
	\item Поиск гонок
	\item Поиск неверной работы с памятью (use-after-free, leak, uninitialized memory access etc)
	\item Поиск undefined behaviour
\end{itemize}

\end{frame}

\begin{frame}[t, noframenumbering]{Существующие подходы к описанию моделей памяти}
\framesubtitle{Прагматичные подходы: дай человеку спиннинг}

C/С++ до 11 версии + POSIX threads + санитайзеры

Динамические проверки свойств программы:
\begin{itemize}
	\item Поиск гонок
	\item Поиск неверной работы с памятью (use-after-free, leak, uninitialized memory access etc)
	\item Поиск undefined behaviour
\end{itemize}

Valgrind\footnote{\tiny\url{https://valgrind.org/}}

LLVM sanitizers\footnote{\tiny\url{https://github.com/google/sanitizers}}

\end{frame}


\begin{frame}{Существующие подходы к описанию моделей памяти}
\framesubtitle{Прагматичные подходы: разрешенное подмножество операций}

Специфицировать подмножество операций \textbf{языка}, предназначенных для многопоточных программ.

\begin{itemize}
\pause
\item C/C++ atomics

\pause
\item Swift/ObjC NSLocking

\pause
\item Java-1996 volatile
\end{itemize}

\pause
Не путайте с реализацией на уровне библиотеки pthreads!

\pause
Не так просто сделать: "The Java Memory Model is Fatally Flawed"\ , William Pugh, 2000\footnote<6->{\tiny\url{http://www.cs.umd.edu/~pugh/java/broken.pdf}}

\end{frame}


\begin{frame}{Существующие подходы к описанию моделей памяти}
\framesubtitle{Продвинутые подходы: исчерпывающая модель памяти}

\pause
С использованием всего доступного арсенала: 
\begin{itemize}
	\pause
	\item An action \textit{a} is described by a tuple $\langle t, k, v, u\rangle$ comprising ...\
	\pause
	\item Частичные, линейные порядки; транзитивное замыкание бинарных отношений; happens-before
	\pause
	\item Causality requirements, circular hp, out-of-thin-air problem
	\pause
	\item Adaptation to h/w models\footnote<6->{\tiny"JSR-133 Cookbook for Compiler Writers"\ \url{https://gee.cs.oswego.edu/dl/jmm/cookbook.html}}
	\pause
	\item Каждый data race имеет разрешенные и запрещенные последствия
\end{itemize}

\pause
Подход обладает рядом недостатков:
\begin{itemize}
	\item Очень сложно, долго и дорого.
	\item Будут недочеты\footnote<8->{\tiny"Java Memory Model Examples: Good, Bad and Ugly"\ \url{https://groups.inf.ed.ac.uk/request/jmmexamples.pdf}}.
	\item Мало кто в мире будет в состоянии полностью понять написанное. Еще меньше людей смогут применить на практике.
\end{itemize}
\end{frame}


\begin{frame}[t]{Java Memory Model}
Java language specification: \url{https://docs.oracle.com/javase/specs/}

\pause
Глава 17 "Threads and Locks"

\pause
Раздел 17.4 Memory Model (15 страниц)

\pause
Основная идея -- давайте попытаемся ввести частичный порядок среди различных событий: операции с полями, создание потоков, исполнение synchronized.
Это поможет нам рассуждать о том, что было "раньше"\ либо "позже".

\end{frame}


\begin{frame}[t,noframenumbering]{Java Memory Model}
Java language specification: \url{https://docs.oracle.com/javase/specs/}

Глава 17 "Threads and Locks"

Раздел 17.4 Memory Model (15 страниц)

Основная идея -- давайте попытаемся ввести частичный порядок среди различных событий: операции с полями, создание потоков, исполнение synchronized.
Это поможет нам рассуждать о том, что было "раньше"\ либо "позже".


Попытки рассказать просто о сложном от Алексея Шипилева\footnote{\tinyДля любителей выбирать между слайдами, конспектом и видео: \url{https://shipilev.net}}: \url{https://shipilev.net/talks/geecon-May2018-jmm.pdf}. Обратите внимание на слайд "Further Reading".

\end{frame}


\begin{frame}{Существующие подходы к описанию моделей памяти}
\framesubtitle{Продвинутые подходы: исчерпывающая модель памяти + паттерны}

Doug Lea, private communication with Aleksey Shipilev, 2013\footnote{\tiny Citation from \url{https://shipilev.net/blog/2014/jmm-pragmatics}, slide 109}

\begin{quote}
The best way to build up a small repertoire of constructions that you know the answers for and then never think about the JMM rules again unless you are forced to do so! Literally nobody likes figuring things out from the JMM rules as stated, or can even routinely do so correctly.
This is one of the many reasons we need to overhaul JMM someday.
\end{quote}
\end{frame}


\begin{frame}{Существующие подходы к описанию моделей памяти}
\framesubtitle{Продвинутые подходы: исчерпывающая модель памяти + паттерны}

Нам понадобятся:
\begin{itemize}
	\item Полная и исчерпывающая модель памяти языка
	\item Набор заведомо корректных и легко запоминаемых шаблонов многопоточного программирования
\end{itemize}

\pause
В Java неувядающей классикой считается шаблон "Double checked locking"\footnote<2->{\tiny"Double-Checked Locking is Broken"\ \url{http://www.cs.umd.edu/~pugh/java/memoryModel/DoubleCheckedLocking.html}}.
\pause
Который, если аккуратно его реализовать, в современной Java корректен. В некоторых командах вполне заслуженно считается антипаттерном/опасным кодом, т.к. существуют альтернативы\footnote<3->{\tiny\url{https://en.wikipedia.org/wiki/Initialization-on-demand_holder_idiom}}.
 
\pause
Рекомендую замечательные книги "Java Concurrency in Practice"\ , "Effective Java"\ и прекрасную документацию к пакету \texttt{java.util.concurrent}.


\end{frame}

\begin{frame}[t]{Существующие подходы к описанию моделей памяти}
\framesubtitle{Продвинутые подходы: исчерпывающая модель памяти + паттерны}

\pause
Недостатки подхода:

\pause
 \begin{itemize}
 	\item Недостатки? Какие недостатки?
 	\pause
 	\item Вы же на курсе по Java, а это самый лучший язык программирования :)
 \end{itemize}

\end{frame}


\begin{frame}[fragile,t,noframenumbering]{Существующие подходы к описанию моделей памяти}
\framesubtitle{Продвинутые подходы: исчерпывающая модель памяти + паттерны}

Многопоточность сама по себе всё еще остается весьма сложной сущностью.
\begin{itemize}
	\item Java Concurrency in Practice -- 403 страницы
	\item The Art of Multiprocessor Programming -- 508 страниц
	\item Is Parallel Programming Hard, And, If So, What Can You Do About It? -- 634 страницы
\end{itemize}

\pause
\begin{quote}
 ... во многой мудрости много печали; и кто умножает познания, умножает скорбь.
\end{quote}
\end{frame}


\begin{frame}[fragile,t,noframenumbering]{Существующие подходы к описанию моделей памяти}
\framesubtitle{Продвинутые подходы: исчерпывающая модель памяти + паттерны}

Многопоточность сама по себе всё еще остается {\color{red} интересной и непонятной штукой}.
\begin{itemize}
	\item Java Concurrency in Practice -- 403 страницы
	\item The Art of Multiprocessor Programming -- 508 страниц
	\item Is Parallel Programming Hard, And, If So, What Can You Do About It? -- 634 страницы
\end{itemize}

\end{frame}



\begin{frame}{Существующие подходы к описанию моделей памяти}
\framesubtitle{А что же плюс\'{ы}???}

\pause
Модель памяти для С++11 разрабатывалась с учетом опыта и ошибок Java Memory Model. Во многом -- теми же людьми.

\pause
Но у C++ есть свой "багаж"\ -- undefined behaviour.

\pause
С одной стороны, это развязывает руки писателям спецификации -- не нужно описывать все-все краевые случаи (жизненно важно для управляемого безопасного языка).

\pause
С другой стороны, инструментов для контроля всего и вся в языке гораздо больше (volatile, atomics, inline assembly, std::thread ...).   

\pause
Мне сложно кратко и по существу рассказать про такой монументальный артефакт человеческого труда как "C++11"\ , поэтому просто посоветую почитать заметку от
Russ Cox\footnote<6->{\tiny\url{https://research.swtch.com/plmm}}.

\pause
Кратко: кое-какие дизайн решения некоторые уважаемые люди считают как минимум спорными.

\end{frame}


\begin{frame}{Существующие подходы к описанию моделей памяти}
\framesubtitle{Продвинутые подходы: ahead-of-time checks}

\begin{itemize}
	\pause
	\item Сделать \textbf{некоторые} некорректные многопоточные программы некомпилируемыми (Rust)
	\pause
	\item Выразить свойства программы в типах, вывести эти свойства из исходного текста (сепарационные логики, языки с зависимыми типами, инструменты дедуктивной верификации программ)
	\pause
	\item Проверка моделей (model checking)
\end{itemize}

\pause
Особенности:
\begin{itemize}
	\item Традиционно считается, что требуется б\'{о}льшая квалификация.
	\item Иногда подход героически решает проблемы, которых и так нет в управляемых языках.
	\item Понимаете достоинства и недостатки таких инструментов -- сумеете написать корректные программы почти на любом языке программирования.
\end{itemize}
\end{frame}


\begin{frame}[t]{Существующие подходы к описанию моделей памяти}
\framesubtitle{Current research: hardware-level transactions}

"Как перестать бояться data-race и начать жить?"

\pause

\begin{itemize}
	\item Игнорировать -- некорректно

	\pause
	\item Запретить -- undefined behaviour hell (C++)

	\pause
	\item Детектировать -- проблема "spurious sanitizer fail"

	\pause
	\item Объяснить -- может быть весьма контринтуитивно (Java)

	\pause
	\item Не допускать -- страдать от deadlock-ов и т.п.
\end{itemize}

\end{frame}


\begin{frame}[t,noframenumbering]{Существующие подходы к описанию моделей памяти}
\framesubtitle{Current research: hardware-level transactions}

"Как перестать бояться data-race и начать жить?"

\begin{itemize}
	\item Игнорировать -- некорректно
	\item Запретить -- undefined behaviour hell (C++)
	\item Детектировать -- проблема "spurious sanitizer fail"
	\item Объяснить -- может быть весьма контринтуитивно (Java)
	
	\item \sout{Не допускать -- страдать от deadlock-ов и т.п.}

	\pause
	\item Не допускать -- "ляг, поспи, и всё пройдет"
\end{itemize}

\pause
Hardware and Software transactional memory\footnote<3->{\tiny\url{https://en.wikipedia.org/wiki/Transactional_memory}}

\pause
Пока остается экспериментальной технологией.

\end{frame}


\begin{frame}{Существующие подходы к описанию моделей памяти}
\framesubtitle{Current research: program synthesis}

Генерировать программы по описанию алгоритма, корректные по построению.

\pause
Нет, я не предлагаю использовать ChatGPT.

\pause
Синтез программ -- это интересно!
\begin{itemize}
	\item Constraint solvers
	\item Fuzzers
	\item Evolutionary algorithms
	\item Conformance testing
	\item Program verification
\end{itemize}

\pause
Пока -- не очень доступно:
\begin{itemize}
	\item Концептуальная сложность написания спецификаций
	\item Вычислительная сложность поиска подходящей программы
	\item Экономическая сложность внедрения и поддержки
	\item Фундаментальная сложность проверки свойств произвольной программы
\end{itemize}

\end{frame}

\begin{frame}{Существующие подходы к описанию моделей памяти}
\framesubtitle{Подходы следующего поколения}

\pause
 
 Может быть именно вы станете их автором

 % RustBelt Meets Relaxed Memory
 % Finding and Fixing a Mismatch Between the Go Memory Model and Data-Race Detector: A Story on Applied Formal Methods
 % TODO moiseenko pdkopae, lmm comparison + promising semantics for langs
\end{frame}




% \begin{frame}{Иллюстрирующий пример}
% Пример
% с IRIW
% Есть порядок но не между всеми операциями. Он частичный. Допусакает несколько суенариев исполнение. Но не все.
% MM -- задает множество допустимых сценариев исполнения
% \end{frame}
% 
% 
% \begin{frame}{Ключевое слово synchronized}
% Иногда вы хотите линейный порядок. Для этого есть два основных иснтсрумена
% - synchronized. Все блоки внутри будут упорядочены (для данного монитора)
% Пример independent monitors
% \end{frame}
% 
% \begin{frame}{Ключевое слово volatile}
% Слишком крупно
% 
% - volatile, порядок на read/write полей.
% Roach motel, шаблон с публикацией/инициализацией
% \end{frame}
% 
% 
% \begin{frame}{illustration with doubel checked locking}
% но вообще есть более надежные инструменты типа singleton clinit и т.п.
% \end{frame}


\begin{frame}{Луч надежды}

Вместо чтения фундаментальных трудов по многопоточному программированию, слабым моделям памяти и спецификациям процессоров, достаточно
\begin{itemize}
	\pause
	\item выбрать \textit{правильный} язык программирования

	\pause
	\item прочитать короткий сборник советов о том, как пользоваться \texttt{java.util.concurrent}

	\pause
	\item освоить пару распространенных Java-специфичных шаблонов написания потокобезопасного кода 
\end{itemize}

\pause
И можно приступать к написанию production кода.

\pause
Экономия времени прикладных программистов ценой увеличенных издержек дизайнеров языка.

\end{frame}


\begin{frame}{Предостережение}

Помните! 

\pause
Все алгоритмические проблемы многопоточности (deadlock, livelock, starvation, lock convoy, priority inversion и т.п.) всё еще представляют угрозу.

\pause
Берегитесь! 

\pause
Очень мало людей \textbf{действительно} хорошо понимают современные модели памяти. Если вам приходится часто рассуждать про свой код в стиле "тут случается happens-before между двумя доступами к памяти, а потом через intra-thread order мы протягиваем зависимость до той volatile операции, чтобы в итоге ..."\ -- это ОЧЕНЬ тревожный звоночек. 

\pause
С вероятностью 99\% вы себя вводите в заблуждение. 

\pause
Уверены в корректности? Задумайтесь о поддерживаемости кода. Гораздо лучше, чтобы программа была собрана из понятных, широко известных и проверенных шаблонов-кирпичиков

\end{frame}

% TODO On the Validity of Program Transformations in the Java Memory Model” (2007).)
% (Adve and Boehm, “Memory Models: A Case For Rethinking Parallel Languages and Hardware,” August 2010) 
%  “Common Compiler Optimisations are Invalid in the C11 Memory Model and what we can do about it” (2015),
% The Problem of Programming Language Concurrency Semantics”
% “Constructing a Weak Memory Model”

% Twenty-five years after the first Java memory model, and after many person-centuries of research effort, we may be starting to be able to formalize entire memory models. Perhaps, one day, we will also fully understand them. 


% // kernel mm: failed lock do not imply any sort of barrier
% // acquire+release are not full barrier for LOCKS !!! (wrt other cpu not holding lock)


% TODO; шутка про то что volatile в Java != acquire-release в C++ это сематически две разных вещи. ЧТо же имеют в виду сишарперы?

 


\section{Подведение итогов}

\begin{frame}{Заключение}

\begin{itemize}
    \item Современные компиляторы -- это сложно
    \item Современные процессоры -- это сложно
\end{itemize}

\pause
Следствие -- спецификация современного многопоточного высокопроизводительного надежного языка довольно сложна. 

\pause
К счастью, обычно авторы желают облегчить жизнь разработчикам и предоставляют рецепты/шаблоны/паттерны, которые надежно работают. Вы можете их заучить и применять.

\pause
Иногда этого недостаточно, например, если вы
\begin{itemize}
    \pause
    \item разработчик продвинутых многопоточных алгоритмов
    \pause
    \item автор оптимизирующего компилятора или рантайма
    \pause
    \item инженер по производительности, выжимающий максимальную скорость ценой хрупкого кода\footnote<7->{\tinyКрасная зона в терминах \url{https://youtu.be/p2b4JHESEOc}}
    \pause
    \item исследователь корректности и надежности concurrent software 
    \pause
    \item любите понимать происходящее на глубоком уровне
\end{itemize}
\end{frame}

\begin{frame}{Почитать}
Книги
\begin{itemize}
    \item "The Art of Multiprocessor Programming"\ by M. Herlihy \& N. Shavit
    \item "Is Parallel Programming Hard, And, If So, What Can You Do About It?"\ by Paul E. McKenney
    \item "Java Concurrency in Practice"\ by Brian Goetz et al.
\end{itemize}

Статьи
\begin{itemize}
    \item "Memory Models"\ series by Russ Cox\footnote{\tiny\url{https://research.swtch.com/mm}}
    \item "Threads Cannot be Implemented as a Library"\ by Hans-J. Boehm
    \item "A Tutorial Introduction to the ARM and POWER Relaxed Memory Models"\ by L. Maranget et al.
    \item "Memory Barriers: a Hardware View for Software Hackers"\ by Paul E. McKenney
\end{itemize}
\end{frame}

\begin{frame}[noframenumbering]{Посмотреть}
\begin{itemize}
    \item Роман Елизаров, "Многопоточное программирование — теория и практика"\ \url{https://youtu.be/mf4lC6TpclM}
    \item Алексей Шипилев, JMM series \url{https://shipilev.net}
    \item Herb Sutter, C++ and Beyond 2012, "Atomic Weapons"\ series \url{https://youtu.be/A8eCGOqgvH4}
\end{itemize}
\end{frame}

\begin{frame}
Q \& A
\end{frame}

\end{document}
