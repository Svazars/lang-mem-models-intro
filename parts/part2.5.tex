
\section{Программист хочет как лучше...}
\showTOC

\begin{frame}[t]{Проблема №1}

Компилятор, в погоне за производительностью, начинает "фантазировать":
\begin{itemize}
	\pause
	\item Добавлять операции, которых не было в исходной программе
	\item Удалять написанные в исходном тексте операции
	\item Изменять порядок операций
\end{itemize}

\pause
Программисту хотелось бы \only<3>{явно сказать "делай в точности как написано"}\only<4->{иметь языковые средства, позволяющие контролировать происходящее}.
\pause

\end{frame}

\begin{frame}[t, fragile]{Compiler barriers}

\vspace{-0.5cm}
\begin{tabular}{p{0.5\textwidth} p{0.5\textwidth}}
\begin{minted}[fontsize=\small]{c}
int x, y;
void foo() {
      x = 1;
      y = 2;
      x = 3;
}
\end{minted}
&
\end{tabular}

\vspace{-0.5cm}

\pause

\begin{tabular}{p{0.5\textwidth} p{0.5\textwidth}}

\begin{minted}[fontsize=\small]{gas}
foo1:
    mov [y], 2
    mov [x], 3
    ret
\end{minted}

& 

\begin{minted}[fontsize=\small]{gas}
foo2:
    mov [x], 1
    mov [y], 2
    mov [x], 3
    ret
\end{minted}

\end{tabular}

\pause
Современный оптимизирующий компилятор выберет вариант слева\footnote<3->{\tiny\url{https://godbolt.org/z/r6sMzrj6K}}.

\pause
Но можно ему сказать -- "вот эта точка в программе -- барьер, она запрещает двигать операции через него".

\end{frame}


\begin{frame}[t, fragile]{Compiler barriers}

\vspace{-0.5cm}
\begin{tabular}{p{0.5\textwidth} p{0.5\textwidth}}
\begin{minted}[fontsize=\small]{c}
int x, y;
void foo1() {
      x = 1;
      y = 2;
      x = 3;
}
\end{minted}

& 

\begin{minted}[fontsize=\small]{c}
int x, y;
void foo2() {
      x = 1;
      barrier();
      y = 2;
      x = 3;
}
\end{minted}
\end{tabular}

\pause
\vspace{-0.5cm}

\begin{tabular}{p{0.5\textwidth} p{0.5\textwidth}}

\begin{minted}[fontsize=\small]{gas}
foo1:
    mov [y], 2
    mov [x], 3
    ret
\end{minted}

& 

\begin{minted}[fontsize=\small]{gas}
foo2:
    mov [x], 1
    mov [y], 2
    mov [x], 3
    ret
\end{minted}
\end{tabular}

\pause
Обратите внимание, что это ОЧЕНЬ низкоуровневый механизм и при написании современного C/C++ кода его использовать не следует\footnote<3->{\tiny\url{https://preshing.com/20120625/memory-ordering-at-compile-time/}}.

\end{frame}


\begin{frame}{Проблема №2}
Процессор, в погоне за производительностью, начинает "чудить":
\begin{itemize}
	\pause
	\item С разной скоростью передавать информацию другим процессорам 
	\item Изменять порядок операций
\end{itemize}

\pause
Программисту хотелось бы иметь языковые средства, позволяющие контролировать происходящее.

\pause
В многоядерном процессоре предусмотрены специальные инструкции, которые позволяют установить порядок "видимости"\ среди операций чтения и записи\footnote<4->{\tiny\url{https://en.wikipedia.org/wiki/Memory_barrier}}.

\end{frame}

\begin{frame}[fragile]{Memory barriers}
\framesubtitle{Пример}

\vspace{-0.5cm}

\begin{tabular}{p{0.5\textwidth} p{0.5\textwidth}}
\begin{minted}[fontsize=\small]{gas}
# thread A
mov [x] ,  1
mov EAX , [y]
\end{minted}

& 

\begin{minted}[fontsize=\small]{gas}
# thread B          
mov [y] , 1
mov EBX, [x]
\end{minted}
\end{tabular}

Допустимые результаты: 
\pause 
\texttt{(0, 0) (1, 0) (0, 1) (1, 1)}

\pause
\vspace{-0.25cm}

\begin{tabular}{p{0.5\textwidth} p{0.5\textwidth}}
\begin{minted}[fontsize=\small]{gas}
# thread A
mov [x] ,  1
mfence      
mov EAX , [y]
\end{minted}

& 

\begin{minted}[fontsize=\small]{gas}
# thread B          
mov [y] , 1
mfence
mov EBX, [x]
\end{minted}
\end{tabular}

\pause
Инструкция \texttt{mfence} отменяет переупорядочивание операций с памятью на уровне процессора\footnote<4->{\tiny\url{https://www.felixcloutier.com/x86/mfence.html}}.
-
\pause

Допустимые результаты: \pause \texttt{(1, 0) (0, 1) (1, 1)}
\end{frame}


\begin{frame}{Memory barriers}
\framesubtitle{Предостережение}

Не забывайте
\begin{itemize}
  \pause
  \item Барьеры дорогие

  \pause
  \item Инструкции обладают процессорно-специфичной и иногда весьма запутанной семантикой

  \pause
  \item Расстановка барьеров в ряде многопоточных алгоритмов -- не единственная и, по сути, является смесью искусства и инженерного мастерства

  \pause
  \item Практически все, кто так делает, похожи на глотателей огня. Даже если живы, то со шрамами от ожогов\footnote<5->{\tiny\url{https://www.researchgate.net/publication/228824849_Memory_Barriers_a_Hardware_View_for_Software_Hackers}}.
\end{itemize}
\end{frame}


\begin{frame}[t]{Языковые средства}
\framesubtitle{Вопросы взаимного влияния}

Краткая выжимка из предыдущих слайдов. Для "починки"\ многопоточных алгоритмов бывает необходимо:
\begin{itemize}
  \item запрещать компилятору "фантазировать"
  \item запрещать процессору "чудить"
\end{itemize}

\pause
Например, с помощью барьеров разного вида. 

\pause
Кажется, что компиляторные и процессорные барьеры -- это разные сущности и, наверное, независимые друг от друга. 

\pause
Должен ли компилятор уважать написанные программистом в исходном коде процессорные барьеры?

\pause
Должен ли процессор уважать написанные программистом в исходном коде компиляторные барьеры?
\end{frame}

\begin{frame}[t, noframenumbering]{Языковые средства}
\framesubtitle{Вопросы взаимного влияния}

Краткая выжимка из предыдущих слайдов. Для "починки"\ многопоточных алгоритмов бывает необходимо:
\begin{itemize}
  \item запрещать компилятору "фантазировать"
  \item запрещать процессору "чудить"
\end{itemize}

Нужен ли программисту независимый контроль над обеими проблемами, с учетом того, что допустить ошибку невероятно легко, а негативные последствия практически невозможно отладить?

\end{frame}


\begin{frame}[t]{Языковые средства}
\framesubtitle{Два мира}

Краткая выжимка из предыдущих слайдов. Для "починки"\ многопоточных алгоритмов бывает необходимо:
\begin{itemize}
  \item запрещать компилятору "фантазировать"
  \item запрещать процессору "чудить"
\end{itemize}

Существуют разные подходы:
\begin{itemize}
  
  \pause
  \item Языки, с маниакальной страстью пытающиеся дать разработчикам способы написать очень быструю, но некорректную программу.
  \pause
  C/C++

  \pause
  \item Языки, сфокусированные на безопасности и с помощью управляемой среды создающие "песочницу".
  \pause
  Java
\end{itemize}
\end{frame}


\begin{frame}[t,noframenumbering]{Языковые средства}
\framesubtitle{Два мира}

Краткая выжимка из предыдущих слайдов. Для "починки"\ многопоточных алгоритмов бывает необходимо:
\begin{itemize}
  \item запрещать компилятору "фантазировать"
  \item запрещать процессору "чудить"
\end{itemize}

Существуют разные подходы:
\begin{itemize}
  
  \item Языки,\only<1-2>{ с маниакальной страстью пытающиеся дать разработчикам способы написать очень быструю, но некорректную программу}\only<3>{ \sout{с маниакальной страстью пытающиеся дать разработчикам способы написать очень быструю, но некорректную программу}}\only<4->{ {\color{red}предоставляющие разработчикам документированное и низкоуровневое API для написания очень быстрых программ}}.
  C/C++

  \item Языки, сфокусированные на безопасности и с помощью управляемой среды\only<1>{ \sout{создающие}}\only<2->{ {\color{red}пытающиеся создать}} "песочницу".
  Java
\end{itemize}

\end{frame}


\begin{frame}[fragile, t]{Разные миры}
\framesubtitle{Пример}

\begin{minted}[fontsize=\small]{c}
static volatile int x, y;
\end{minted}

\begin{tabular}{p{0.5\textwidth} p{0.5\textwidth}}

\begin{minted}[fontsize=\small]{c}
void threadA() {
      x = 1;
  int a = y;
}
\end{minted}

& 

\begin{minted}[fontsize=\small]{c}
void threadB() {                                   
        y = 1;                           
    int b = x;                           
}                    
\end{minted}
\end{tabular}

\pause
\vspace{-0.5cm}
C\footnote<2->{\tiny\url{https://godbolt.org/z/q4raxqrTe}}:

\begin{tabular}{p{0.5\textwidth} p{0.5\textwidth}}
\begin{minted}[fontsize=\small]{gas}
mov [x] ,  1
mov EAX , [y]
\end{minted}

& 

\begin{minted}[fontsize=\small]{gas}
mov [y] , 1
mov EBX, [x]
\end{minted}
\end{tabular}

\pause
Допустимые результаты: \texttt{(0, 0) (1, 0) (0, 1) (1, 1)}
\end{frame}

\begin{frame}[fragile, t, noframenumbering]{Разные миры}
\framesubtitle{Пример}

\begin{minted}[fontsize=\small]{c}
static volatile int x, y; 
\end{minted}

\begin{tabular}{p{0.5\textwidth} p{0.5\textwidth}}

\begin{minted}[fontsize=\small]{c}
void threadA() {
      x = 1;
  int a = y;
}
\end{minted}

& 

\begin{minted}[fontsize=\small]{c}
void threadB() {                                   
        y = 1;                           
    int b = x;                           
}                    
\end{minted}
\end{tabular}

\vspace{-0.5cm}
Java\footnote{\tiny\url{https://github.com/Svazars/lang-mem-models-intro/tree/main/samples/java}}:


\begin{tabular}{p{0.5\textwidth} p{0.5\textwidth}}
\begin{minted}[fontsize=\small]{gas}
mov [x] ,  1
lock add [rsp], 0x0
mov EAX , [y]
\end{minted}

& 

\begin{minted}[fontsize=\small]{gas}
mov [y] , 1
lock add [rsp], 0x0
mov EBX, [x]
\end{minted}
\end{tabular}

\pause
\vspace{-0.5cm}
\texttt{lock add [rsp], 0x0} $\approx$ \texttt{mfence} $\approx$ full memory barrier\footnote<2->{\tiny\url{https://shipilev.net/blog/2014/on-the-fence-with-dependencies/}}

\pause
Допустимые результаты: \texttt{(1, 0) (0, 1) (1, 1)}

\end{frame}


\begin{frame}[fragile, t, noframenumbering]{Разные миры}
\framesubtitle{Пример}

\begin{minted}[fontsize=\small]{c}
static volatile int x, y; 
\end{minted}

\begin{tabular}{p{0.5\textwidth} p{0.5\textwidth}}
\begin{minted}[fontsize=\small]{c}
void threadA() {
      x = 1;
  int a = y;
}
\end{minted}

& 

\begin{minted}[fontsize=\small]{c}
void threadB() {                                   
        y = 1;                           
    int b = x;                           
}                    
\end{minted}
\end{tabular}

Допустимые результаты в C: \texttt{(0, 0) (1, 0) (0, 1) (1, 1)}

Допустимые результаты в Java: \texttt{(1, 0) (0, 1) (1, 1)}

\pause
Обычно пишут, что \texttt{volatile} в языке Си ограничивает только преобразования на уровне компилятора, а в языке Java -- ограничивает как компилятор, так процессор.

\pause
Но давать слишком простую модель реального мира -- плохо с педагогической точки зрения.

\pause
Поэтому защищу себя от гнева ваших будущих работодателей.

\end{frame}


\begin{frame}{Совет от мудрой совы}

В языке С вместо \texttt{volatile} \textbf{всегда} используйте типы из \texttt{stdatomic.h}\footnote{\tiny\url{https://en.cppreference.com/w/c/language/atomic}} для разделяемых переменных.

\pause

Если стандарт C11 не поддерживается используемым компилятором или в коде обычные переменные изменяются из разных потоков одновременно -- бегите\footnote<2->{\tiny\url{https://www.kernel.org/doc/Documentation/process/volatile-considered-harmful.rst}}.

\pause

Для любителей почитать как ругается Линус Торвальдс, советую обратить внимание на цепочку обсуждений "volatile considered evil"\footnote<3->{\tiny\url{https://lkml.org/lkml/2006/7/6/159}}:

\begin{quote}
 "volatile"\ really \_is\_ misdesigned. The semantics of it are so unclear as to be totally useless. The only thing "volatile"\ can ever do is generate worse code, WITH NO UPSIDES.
\end{quote}

\end{frame}


\begin{frame}[fragile]{Языковые средства написания работающих многопоточных программ}
\framesubtitle{Выводы}

\begin{itemize}

 \pause
 \item Пожалуйста, никогда и никому не говорите что \texttt{volatile} в С и в Java имеют один и тот же смысл
 
 \pause
 \item Не думайте, что бездумное добавление \textt{volatile} в вашу программу сделает её корректной (даже на Java это не так)
 
 \pause
 \item Всегда помните, что в разных языках "одинаковая"\ конструкция может иметь весьма разный смысл
 \begin{itemize}
    \item смена языкового стека
    \item адаптация алгоритма из библиотеки/публикации
    \item кросс-языковая трансляция
 \end{itemize}

 \pause
 \item Компиляторные и процессорные барьеры -- прямолинейный способ добиться желаемого при написании многопоточных программ

 \pause
 \item Барьерный подход несет с собой много неявной сложности и специфики (языка, компилятора, процессора)
\end{itemize}

\end{frame}

