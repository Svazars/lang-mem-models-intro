
\section{Компилятор хотел как лучше, а получилось...}
\showTOC

\begin{frame}[fragile]{Классические оптимизации однопоточного кода}
\framesubtitle{Inventing reads}

\begin{lstlisting}
static int a;
void foo_1() {
    int tmp;
    while (tmp = a) {
        do_something_with(tmp);
    }
}
\end{lstlisting}

\pause

Имеет ли право компилятор сэкономить регистры и переписать функцию следующим образом?

\begin{lstlisting}
void foo_2() {
    while (a) {
        do_something_with(a);
    }
}
\end{lstlisting}

\pause

Если в программе существуют другие потоки, изменяющие переменную \texttt{a} -- такое преобразование безопасно?
\end{frame}


\begin{frame}[fragile]{Классические оптимизации однопоточного кода}
\framesubtitle{Removing reads}

\begin{lstlisting}
static int a;
void foo_1() {
    int tmp;
    while (tmp = a) {
        do_something_with(tmp);
    }
}
\end{lstlisting}

\pause

Имеет ли право компилятор уменьшить количество загрузок из памяти и переписать функцию следующим образом?

\begin{lstlisting}
void foo_3() {
    int tmp = a;
    if (tmp)
        for (;;) do_something_with(tmp);
}
\end{lstlisting}

\pause

Если в программе существуют другие потоки, изменяющие переменную \texttt{a} -- такое преобразование безопасно?
\end{frame}


\begin{frame}[fragile]{Классические оптимизации однопоточного кода}
\framesubtitle{Godbolt}

\begin{adjustbox}{width=\textwidth,height=1cm,keepaspectratio}
\begin{lstlisting}
static int a;
void foo_1() {
    int tmp;
    while (tmp = a) {
        do_something_with(tmp);
    }
}
\end{lstlisting}
\end{adjustbox}

\texttt{x86-64 clang 16.0.0 -O2}\footnote{\url{https://godbolt.org/z/51W5adoTG}}:

\begin{adjustbox}{width=\textwidth,height=2cm,keepaspectratio}
\begin{lstlisting}
foo_1:                                  
        push    rbx
        mov     ebx, dword ptr [rip + a]
        test    ebx, ebx
        je      .LBB1_2
.LBB1_1:                           #<---------|    
        mov     edi, ebx           #          |
        call    do_something_with  #          |
        jmp     .LBB1_1            #----------|
.LBB1_2:
        pop     rbx
        ret
\end{lstlisting}
\end{adjustbox}

\end{frame}


\begin{frame}{Классические оптимизации однопоточного кода}
\framesubtitle{Конфликт интересов}

\begin{itemize}
    \item Хотим мощный оптимизирующий компилятор, чтобы наш однопоточный код работал как можно быстрее
    \item Не хотим, чтобы преобразования программ ломали наш многопоточный код, который улучшает производительность
\end{itemize}

\pause
Классический пример конфликтующих оптимизаций.

\pause
Как бы вы решали данную проблему?

\pause
\begin{itemize}
 \item Запретить какие-то преобразования (blacklist)
 \item Разрешить только "правильные"\ преобразования (whitelist)
\end{itemize}

\end{frame}

\begin{frame}{Вызовы для авторов языков}

Необходимо понять, какие преобразование "подозрительные"\ и включить нужные ремарки в спецификацию языка.

\pause
\begin{itemize}
 \item Как понять, что запреты исчерпывающие? 
 \item Как проверить, что указания непротиворечивы?
\end{itemize}

\pause
В примере "removing reads"\ используются довольно простые классические преобразования. Компиляторная классика -- CSE, DCE, UCE, loop-invariant code motion.

\pause
Надо написать некоторую спецификацию того, как различные потоки видят изменения разделяемой памяти (переменных, объектов, полей). 

\pause
Соблюсти баланс между \textit{понятностью}, \textit{производительностью} и \textit{устойчивостью}.

\end{frame}

\begin{frame}{Выводы}

\begin{itemize}
    \item Очевидные и верные преобразования однопоточных программ очень часто искажают поведение многопоточного кода, использующего общую память 
    \item Если язык программирования не готов радикально отказаться от общего изменяемого состояния\footnote{\tiny\url{https://en.wikipedia.org/wiki/Actor_model}}\textsuperscript{,}\footnote{\tiny\url{https://en.wikipedia.org/wiki/Communicating_sequential_processes}} или попытался, но не смог\footnote{\tiny\url{https://kotlinlang.org/docs/multiplatform-mobile-concurrency-overview.html#global-state}}, то необходимо определить "границы дозволенного"\ для оптимизаций и не очень сильно удивлять пользователей языка
    \item Language memory model -- описание того, какие есть гарантии у различных потоков приложения при обращении к разделяемым ячейкам памяти (переменным, объектам, полям, элементам массивов ...)
\end{itemize}

\end{frame}
