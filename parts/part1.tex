
\section{Компилятор хотел как лучше, а получилось...}
\showTOC


\begin{frame}[t,fragile]{Классические оптимизации однопоточного кода}
\framesubtitle{Inventing reads}

\begin{minted}[fontsize=\small]{c}
static int a;
void foo_1() {
  while(true) {
    int tmp = a;
    if (tmp == 0) break;
    do_something_with(tmp);
  }
}
\end{minted}

\pause
Имеет ли право компилятор сэкономить регистры и переписать функцию следующим образом?

\begin{minted}[fontsize=\small]{c}
void foo_2() {
  while(true) {
    if (a == 0) break;
    do_something_with(a);
  }
}
\end{minted}

\end{frame}


\begin{frame}[t,fragile]{Классические оптимизации однопоточного кода}
\framesubtitle{Removing reads}

\begin{minted}[fontsize=\small]{c}
static int a;
void foo_1() {
  while(true) {
    int tmp = a;
    if (tmp == 0) break;
    do_something_with(tmp);
  }
}
\end{minted}

\pause

Имеет ли право компилятор уменьшить количество загрузок из памяти и переписать функцию следующим образом?

\begin{minted}[fontsize=\small]{c}
void foo_3() {
  int tmp = a;
  if (tmp != 0)
    while(true) { do_something_with(tmp); }
}
\end{minted}
\end{frame}




\begin{frame}[t, fragile]{Классические оптимизации однопоточного кода}
\framesubtitle{Godbolt}


\begin{tabular}{p{0.4\textwidth} p{0.5\textwidth}}


\begin{minted}[fontsize=\small]{c}
static int a;
void foo_1() {
  while(true) {
    int tmp = a;
    if (tmp == 0) break;
    do_something_with(tmp);
  }
}
\end{minted}

&

\end{tabular}
\end{frame}


\begin{frame}[t, fragile, noframenumbering]{Классические оптимизации однопоточного кода}
\framesubtitle{Godbolt}


\begin{tabular}{p{0.42\textwidth} p{0.5\textwidth}}


\begin{minted}[fontsize=\small]{c}
static int a;
void foo_1() {
  while(true) {
    int tmp = a;
    if (tmp == 0) break;
    do_something_with(tmp);
  }
}
\end{minted}

\texttt{x86-64 clang 16.0.0 -O2}\footnote{\tiny\url{https://godbolt.org/z/99j3erzaE}}

\texttt{x86-64 gcc 13.1 -O2}\footnote{\tiny\url{https://godbolt.org/z/fxzGEo1qf}}

&

\begin{minted}[fontsize=\small]{gas}
foo_1:                                  
    push    rbx
    mov     ebx, dword ptr [rip + a]
    test    ebx, ebx
    je      .LBB1_2
.LBB1_1:                       #<-|    
    mov     edi, ebx           #  |
    call    do_something_with  #  |
    jmp     .LBB1_1            #--|
.LBB1_2:
    pop     rbx
    ret
\end{minted}

\end{tabular}
\end{frame}


\begin{frame}{Классические оптимизации однопоточного кода}
\framesubtitle{Конфликт интересов}

\begin{itemize}
    \item Хотим мощный оптимизирующий компилятор, чтобы наш однопоточный код работал как можно быстрее
    \item Не хотим, чтобы преобразования программ ломали наш многопоточный код, который улучшает производительность
\end{itemize}

\pause
Классический пример конфликтующих оптимизаций.

\pause
Как бы вы решали данную проблему?

\pause
\begin{itemize}
 \item Запретить какие-то преобразования (blacklist)
 \item Разрешить только "правильные"\ преобразования (whitelist)
\end{itemize}

\end{frame}

\begin{frame}{Вызовы для авторов языков}

Необходимо понять, какие преобразования "подозрительные"\ и включить нужные ремарки в спецификацию языка.

\pause
\begin{itemize}
 \item Как понять, что запреты исчерпывающие? 
 \item Как проверить, что указания непротиворечивы?
\end{itemize}

\pause
В примере "removing reads"\ используются довольно простые классические преобразования. Не хотелось бы запрещать слишком много оптимизаций.

\pause
Надо написать некоторую спецификацию того, как различные потоки видят изменения разделяемой памяти (переменных, объектов, полей). 

\pause
Соблюсти баланс между \textit{понятностью}, \textit{производительностью} и \textit{устойчивостью}.

\end{frame}

\begin{frame}{Выводы}

\begin{itemize}
    \item Очевидные и верные преобразования однопоточных программ очень часто искажают поведение многопоточного кода, использующего общую память 

    \pause
    \item Если язык программирования не готов радикально отказаться от общего изменяемого состояния\footnote<3->{\tiny\url{https://en.wikipedia.org/wiki/Actor_model}}\textsuperscript{,}\footnote<3->{\tiny\url{https://en.wikipedia.org/wiki/Communicating_sequential_processes}} или попытался, но не смог\footnote<3->{\tiny\url{https://kotlinlang.org/docs/multiplatform-mobile-concurrency-overview.html#global-state}}, то необходимо определить "границы дозволенного"\ для оптимизаций и не очень сильно удивлять пользователей языка

    \pause
    \item Language memory model -- описание того, какие есть гарантии у различных потоков приложения при обращении к разделяемым ячейкам памяти (переменным, объектам, полям, элементам массивов ...)
\end{itemize}

\end{frame}
