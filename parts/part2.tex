
\section{Процессор хотел как лучше, а получилось...}
\showTOC

\begin{frame}{Кругом враги}

Не только компиляторы (software) пытаются сломать ваше представление об исполнении программы в многопоточном контексте. 

\pause
Есть еще процессор и подсистема памяти (hardware). 

\pause
Которые умеют:
\begin{itemize}
    \pause
    \item Исполнять независимые инструкции одновременно (out-of-order execution)

    \pause
    \item Задействовать одни и те же ресурсы для исполнения логически независимых потоков (hyper-threading)

    \pause
    \item Спекулировать\footnote<6->{\tiny\url{https://en.wikipedia.org/wiki/Spectre_(security_vulnerability)}}\textsuperscript{,}\footnote<6->{\tiny\url{https://en.wikipedia.org/wiki/Meltdown_(security_vulnerability)}}
    \begin{itemize}
        \pause
        \item О предстоящих переходах (branch prediction)

        \pause
        \item О требуемой памяти (cache prefetching)

        \pause
        \item О результате вычислений (speculative execution)

        \pause
        \item И многом другом
    \end{itemize}
\end{itemize}

\end{frame}

\begin{frame}[fragile,t]{x86: Store buffering}


\begin{minted}[fontsize=\small]{c}
                          int x, y;
\end{minted}

\begin{tabular}{p{0.5\textwidth} p{0.5\textwidth}}

\begin{minted}[fontsize=\small]{c}
void threadA() {
      x = 1;
  int a = y;
}
\end{minted}

& 

\begin{minted}[fontsize=\small]{c}
void threadB() {                                   
        y = 1;                           
    int b = x;                           
}                    
\end{minted}
\end{tabular}


\pause



\begin{tabular}{p{0.5\textwidth} p{0.5\textwidth}}
\begin{minted}[fontsize=\small]{gas}
# thread A
MOV [x] ,  1  # (A.1)
MOV EAX , [y] # (A.2)
\end{minted}

& 

\begin{minted}[fontsize=\small]{gas}
# thread B          
MOV [y] , 1  # (B.1) 
MOV EBX, [x] # (B.2) 
\end{minted}
\end{tabular}

\end{frame}


\begin{frame}[fragile,t,noframenumbering]{x86: Store buffering}

\begin{tabular}{p{0.5\textwidth} p{0.5\textwidth}}
\begin{minted}[fontsize=\small]{gas}
# thread A
MOV [x] ,  1  # (A.1)
MOV EAX , [y] # (A.2)
\end{minted}

& 

\begin{minted}[fontsize=\small]{gas}
# thread B          
MOV [y] , 1  # (B.1) 
MOV EBX, [x] # (B.2) 
\end{minted}
\end{tabular}

\pause
Какие значения для \texttt{(EAX EBX)} допустимы?

\texttt{\ \ \ \ \ \ \ \ \ \ \ \ \ \ \ \ \ \ (1 1)\ , (0 1)\ , (1 0)\ , (0 0)}

\pause
Варианты исполнения:
\begin{itemize}
    \item \texttt{A.1 -> A.2 -> B.1 -> B.2}
    \item \texttt{\ \ \ \ \ \ \       B.1 -> A.2 -> B.2}
    \item \texttt{\ \ \ \ \ \ \ \ \ \ \ \ \ \              B.2 -> A.2}
    \item \texttt{B.1 -> A.1 -> A.2 -> B.2}
    \item \texttt{\ \ \ \ \ \ \ \ \ \ \ \ \ \              B.2 -> A.2}
    \item \texttt{\ \ \ \ \ \ \       B.2 -> A.1 -> A.2}
\end{itemize}
\end{frame}

\begin{frame}[fragile,t,noframenumbering]{x86: Store buffering}

\begin{tabular}{p{0.5\textwidth} p{0.5\textwidth}}
\begin{minted}[fontsize=\small]{gas}
# thread A
MOV [x] ,  1  # (A.1)
MOV EAX , [y] # (A.2)
\end{minted}

& 

\begin{minted}[fontsize=\small]{gas}
# thread B          
MOV [y] , 1  # (B.1) 
MOV EBX, [x] # (B.2) 
\end{minted}
\end{tabular}

Какие значения для \texttt{(EAX EBX)} допустимы?

\texttt{\ \ \ \ \ \ \ \ \ \ \ \ \ \ \ \ \ \ (1 1)\ , (0 1)\ , (1 0)\ , (0 0)}

Варианты исполнения:
\begin{itemize}
    \item \texttt{A.1 -> A.2 -> B.1 -> B.2}                            : \texttt{(0, 1)}
    \item \texttt{\ \ \ \ \ \ \       B.1 -> A.2 -> B.2}               : \texttt{(1, 1)}
    \item \texttt{\ \ \ \ \ \ \ \ \ \ \ \ \ \              B.2 -> A.2} : \texttt{(1, 1)}
    \item \texttt{B.1 -> A.1 -> A.2 -> B.2}                            : \texttt{(1, 1)}
    \item \texttt{\ \ \ \ \ \ \ \ \ \ \ \ \ \              B.2 -> A.2} : \texttt{(1, 1)}
    \item \texttt{\ \ \ \ \ \ \       B.2 -> A.1 -> A.2}               : \texttt{(1, 0)}
\end{itemize}
\end{frame}

\begin{frame}[fragile,t,noframenumbering]{x86: Store buffering}

\begin{tabular}{p{0.5\textwidth} p{0.5\textwidth}}
\begin{minted}[fontsize=\small]{gas}
# thread A
MOV [x] ,  1  # (A.1)
MOV EAX , [y] # (A.2)
\end{minted}

& 

\begin{minted}[fontsize=\small]{gas}
# thread B          
MOV [y] , 1  # (B.1) 
MOV EBX, [x] # (B.2) 
\end{minted}
\end{tabular}

Какие значения для \texttt{(EAX EBX)} допустимы?

\texttt{Ответ:\ \ \ \ \ \ \ \ \ \ \ \ (1 1)\ , (0 1)\ , (1 0)}

Варианты исполнения:
\begin{itemize}
    \item \texttt{A.1 -> A.2 -> B.1 -> B.2}                            : \texttt{(0, 1)}
    \item \texttt{\ \ \ \ \ \ \       B.1 -> A.2 -> B.2}               : \texttt{(1, 1)}
    \item \texttt{\ \ \ \ \ \ \ \ \ \ \ \ \ \              B.2 -> A.2} : \texttt{(1, 1)}
    \item \texttt{B.1 -> A.1 -> A.2 -> B.2}                            : \texttt{(1, 1)}
    \item \texttt{\ \ \ \ \ \ \ \ \ \ \ \ \ \              B.2 -> A.2} : \texttt{(1, 1)}
    \item \texttt{\ \ \ \ \ \ \       B.2 -> A.1 -> A.2}               : \texttt{(1, 0)}
\end{itemize}
\end{frame}

\begin{frame}[fragile,t,noframenumbering]{x86: Store buffering}

\begin{tabular}{p{0.5\textwidth} p{0.5\textwidth}}
\begin{minted}[fontsize=\small]{gas}
# thread A
MOV [x] ,  1  # (A.1)
MOV EAX , [y] # (A.2)
\end{minted}

& 

\begin{minted}[fontsize=\small]{gas}
# thread B          
MOV [y] , 1  # (B.1) 
MOV EBX, [x] # (B.2) 
\end{minted}
\end{tabular}

Какие значения для \texttt{(EAX EBX)} допустимы?

\only<1>{\texttt{Ответ:}}\only<2->{\texttt{Правильный ответ:}} \only<1>{\texttt{\ \ \ \ \ \ \ \ \ \ \ }} \texttt{(1 1)\ , (0 1)\ , (1 0)}
\pause
\texttt{, (0 0)}

\pause
Процессор может переупорядочить записи и чтения, если это не нарушает intra-thread order.
\pause
В данном случае изменился наблюдаемый другими процессорами порядок store и load операций\footnote<4->{\tiny\url{https://habr.com/ru/company/JetBrains-education/blog/523298/}}\textsuperscript{,}\footnote<4->{\tiny\url{https://diy.inria.fr/doc/SB.litmus}}\textsuperscript{,}\footnote<4->{\tiny\url{https://www.cl.cam.ac.uk/~pes20/weakmemory/cacm.pdf}}.

\pause
Вывод: порядок инструкций в машинном коде $\neq$ порядок наблюдаемых эффектов этих инструкций.

\end{frame}


\begin{frame}[fragile]{arm64: Independent Reads of Independent Writes}


\begin{tabular}{p{0.2\textwidth} p{0.2\textwidth} p{0.2\textwidth} p{0.2\textwidth}}
\begin{minted}[fontsize=\small]{python}
thread1
  x = 1
\end{minted}

& 

\begin{minted}[fontsize=\small]{python}
thread2
  y = 1
\end{minted}

&

\begin{minted}[fontsize=\small]{python}
thread3
 r1 = x
 r2 = y 
\end{minted}

&

\begin{minted}[fontsize=\small]{python}
thread4
 r3 = y
 r4 = x
\end{minted}
\end{tabular}


\pause
Может ли быть так, что \texttt{(r1 = 1, r2 = 0, r3 = 1, r4 = 0)}?

%\pause
%Могут ли потоки 3 и 4 увидеть изменения в разном порядке?

\pause
При условии, что переупорядочивание чтений не происходит.

\pause
\begin{itemize}
\item На \texttt{x86} или \texttt{x86\_64} (TSO): нет

\pause
\item На \texttt{ARM} или \texttt{POWER}: да\footnote<5->{\tiny A Tutorial Introduction to the ARM and POWER Relaxed Memory Models, section 6.1}
\end{itemize}

\pause
Записи могут "доехать"\ до других процессоров в разном порядке.

\pause
У каждого процессора своя временная шкала и некоторое видение окружающего мира. Возможно, отличающееся от других процессоров.

\pause
Вывод: нельзя рассматривать "запись в ячейку памяти"\ как точку на единой временной шкале\footnote<8->{\tiny The Art of Multiprocessor Programming by Maurice Herlihy \& Nir Shavit, Chapter 3 "Concurrent Objects"}.
\end{frame}


% \begin{frame}{Домашнее задание}
% 
% Постойте, постойте!
% 
% \pause
% На предыдущих слайдах говорится, что ячейки памяти пишутся и читаются разными процессорами без каких-либо разумных гарантий на порядок.
% 
% \pause
% А на курсе по архитектуре компьютера нам рассказывали, что есть специальные алгоритмы когерентности кэшей\footnote<3->{\tiny\url{https://en.wikipedia.org/wiki/Cache_coherence}}\textsuperscript{,}\footnote<3->{\tiny\url{https://en.wikipedia.org/wiki/MESI_protocol}}, которые не допускают рассинхронизации памяти.
% 
% \pause
% Кто-то из уважаемых преподавателей что-то недоговаривает.
% 
% \pause
% И человек около проектора пришел всего на одну лекцию, а потом исчезнет.
% 
% \pause
% Поразмыслите, как так выходит, что когерентность кэшей есть, а единого порядка видимых эффектов над различными ячейками памяти нет.
% \end{frame}


\begin{frame}{Почему так сложно?}

\begin{itemize}
 \item порядок инструкций в машинном коде $\neq$ порядок наблюдаемых эффектов этих инструкций

 \pause
 \item нельзя рассматривать "чтение/запись в ячейку памяти"\ как точку на единой временной шкале

 \pause
 \item у каждого процессора свои правила
\end{itemize}

\pause
Почему вообще хоть кто-то пользуется ARM/POWER/RISC-V и другими процессорами со слабой моделью памяти?

\pause
\begin{itemize}
  \item производительность
  \pause
  \item Производительность!
  \pause
  \item энергосбережение :)
\end{itemize}

\end{frame}

