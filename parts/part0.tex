

\section{Зачем писать многопоточный код?}
\showTOC


\begin{frame}{Зачем писать параллельные программы?}

Зачем в языке программирования давать средства для написания параллельных(parallel)/конкурентных(concurrent) программ?\footnote{\tiny Rob Pike, Concurrency Is Not Parallelism  \url{https://go.dev/blog/waza-talk}}

\pause

Возможные варианты ответа:
\begin{itemize}
 \pause
 \item Так принято еще с 2000-х годов

 \pause
 \item Современное оборудование фантастически многоядерное

 \pause
 \item Необходимо строить сложные системы, ориентированные на огромное число пользователей

 \pause
 \item А как иначе написать 
 \begin{itemize}
    \pause
    \item сервер обработки входящий соединений?
    
    \pause
    \item многоагентную симуляцию?

    \pause
    \item игру-песочницу типа Minecraft?
 \end{itemize}
\end{itemize}

\pause

В подавляющем большинстве случаев, написание параллельного/распределенного/многопоточного кода -- это оптимизация.

\end{frame}


\begin{frame}[fragile]{Альтернативы}
\framesubtitle{Bash}

Задача: подсчитать число слов в текстовом файле\footnote{\tiny\url{https://github.com/Svazars/lang-mem-models-intro/tree/main/samples/bash}}.

\pause
Последовательное исполнение
\begin{lstlisting}
wc 50_000_000.txt

real    0m 4,900s
user    0m 4,212s
sys     0m 0,240s
\end{lstlisting}

\pause 
Параллельное исполнение (2 процесса)
\begin{lstlisting}
{ ( head -n 25000000 50_000_000.txt | wc) & }
{ ( tail -n 25000000 50_000_000.txt | wc) & }
wait

real    0m 2,323s
user    0m 4,576s
sys     0m 1,084s
\end{lstlisting}

\end{frame}

\begin{frame}[fragile]{Альтернативы}
\framesubtitle{make}

Задача: скомпилировать большой проект на языке C.

\pause

\begin{lstlisting}
make -j8
\end{lstlisting}

\pause

Аналогичные инструменты: 
\begin{itemize}
    \item ant/maven (Java)
    \item groovy DSL (Jenkins)
    \item sbt (Scala)
    \item ...
\end{itemize}

\end{frame}

\begin{frame}{Параллелизм уровня процессов}
\framesubtitle{Обсуждение}

\begin{itemize}
 \pause
 \item ОС создана для того, чтобы быстро, эффективно и надежно управлять процессами
 \pause
 \item Независимые шаги вычислений можно выполнять в разных процессах
 \pause
 \item Write programs that do one thing and do it well\footnote<4->{\tiny\url{https://en.wikipedia.org/wiki/Unix_philosophy}}
\end{itemize}

\pause
Исполнение нескольких потоков вычислений внутри одного процесса не требуется.

\pause 
Параллелизм уровня потоков не нужен.

\pause
Многопоточность не нужна.

\pause
В чем недостатки подхода с использованием только процессов?

\end{frame}


\begin{frame}{Параллелизм уровня процессов}
\framesubtitle{Недостатки}
Процесс -- это идентифицируемая абстракция совокупности взаимосвязанных системных ресурсов на основе отдельного и {\color<8>[rgb]{1,0,0}{независимого виртуального адресного пространства}}\footnote{\tiny{https://ru.wikipedia.org/wiki/Процесс\_(информатика)}}.

\pause

Межпроцессное взаимодействие бывает: %TODO insert бывает, сказать что бывает и хорошо
\begin{itemize}
  \pause
  \item Долгое (latency)

  \pause
  \item Дорогое (throughput)

  \pause
  \item Не всегда кросс-платформенное (Windows/POSIX API vs. protobuf)\footnote<5->{\tiny{https://stackoverflow.com/questions/60649/cross-platform-ipc}}

  \pause
  \item Не всегда надежное (разрыв соединения, смерть процесса)

  \pause
  \item Не всегда безопасное
\end{itemize}

\pause

% TODO: давайте пожертвуем независимостью, вдруг будет лучше
\end{frame}


\begin{frame}{Параллелизм уровня потоков}

Взаимодействие потоков исполнения (threads) внутри общего адресного пространства.

\pause
У каждого потока есть

\begin{itemize}
    \item машинный стек
    \item идентификатор
    \item обработчик сигналов
    \item приоритет в планировщике задач
    \item ...
    \pause
    \item быстрый доступ к общей разделяемой памяти
\end{itemize}

\end{frame}


\begin{frame}{Параллелизм уровня потоков}
\framesubtitle{Применения}

\begin{itemize}
 \pause 
 \item Web-сервер. Клиент шлет запросы, получает ответы, но одна система допускает одновременную обработку нескольких соединений.

 \pause
 \item Многопользовательские игры. Сервер Minecraft позволяет различным игрокам одновременно взаимодействовать с окружением.


 \pause 
 \item Сборщик мусора. Вы создали объект, попользовались им и забыли. Не задумываясь о том, что где-то там есть сборщик мусора, которые отследит ненужность объекта и переиспользует память под следующий объект.
 
  \begin{itemize}
    \pause
    \item Независимые задачи (parallel marking, parallel sweeping)
    
    \pause
    \item Одновременные задачи (concurrent marking, concurrent copying)
    
    \pause
    \item Работоспособность при наличии блокирующих вызовов (поток завис в \texttt{epoll\_wait}, а мусор всё равно был собран)
  \end{itemize}

\pause
 
 \item ...

\end{itemize}
 
\end{frame}


\begin{frame}{Параллелизм уровня потоков}
\framesubtitle{Выводы}

  Весьма хорошая и актуальная оптимизация.

  \pause
  Способ добиться одновременного исполнения в рамках одного процесса.

  \begin{itemize}
    \pause
    \item Всё виртуальное адресное пространство -- общий ресурс

    \pause
    \item Все CPU -- общий ресурс

    \pause
    \item Быстрая скорость обмена информацией

    \pause
    \item Возможность совместно \only<1-8>{использовать}\only<9>{{\color{red} изменять}} данные
  \end{itemize}

  \pause
  Есть общий ресурс -- есть проблемы.

  Deadlock, livelock, starvation, priority inversion, lock convoy, thundering herd problem, ABA problem, use-after-free ...

  \pause
  Сегодня не об этом.

  \pause

\end{frame}




